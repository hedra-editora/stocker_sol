%!TEX root=LIVRO.tex
\chapterspecial{O Gigante Invisível}{}{}
 

O tempo segue em frente no País Sob o Pôr do Sol tanto quanto aqui.

Muitos anos se passaram, e acarretaram muitas mudanças. E, agora,
encontramo"-nos numa época em que as pessoas que viveram no tempo do bom
Rei Mago dificilmente reconheceriam seu próprio Reino se o vissem
novamente.

Tristemente, ele havia mesmo mudado. Não havia mais o mesmo amor ou a
mesma reverência em relação ao rei -- não havia mais a paz perfeita. As
pessoas haviam se tornado mais egoístas e mais gananciosas, e tentavam
tomar tudo o que podiam para si mesmas. Alguns poucos eram muito ricos,
e havia muitos pobres. A~maioria dos belos jardins havia sido devastada.
Casas haviam sido erguidas bem em volta do palácio, e em algumas delas
viviam muitas pessoas que só podiam pagar por parte dela.

Tristemente, todo o belo país estava mudado. O~povo tinha quase se
esquecido do Príncipe Zaphir, que morrera há muitos, muitos anos; e
rosas não mais foram espalhadas pelos caminhos. Aqueles que viviam agora
no País Sob o Pôr do Sol riam da ideia de outros Gigantes, e não os
temiam porque não os tinham visto. Alguns deles diziam:

``Ora! O que há para temer? Mesmo que em algum momento tenham existido
gigantes, eles já não existem mais.''

E assim as pessoas cantavam e dançavam e banqueteavam como antes, e
pensavam somente em si mesmas. Os Espíritos que guardavam o Reino
estavam muito, muito tristes. Suas asas, grandes ao ponto de lançar
sombras, definhavam enquanto eles permaneciam em seus postos nos Portais
do Reino. Os Espíritos escondiam os rostos e tinham os olhos turvos pelo
choro constante, de modo que não notavam se alguma coisa má passava
por eles. Tentaram fazer com que o povo pensasse nos próprios atos
maléficos, mas não podiam deixar seus postos, e as pessoas, ao ouvir
seus lamentos nessa época de trevas, diziam:

``Ouça o suspirar da brisa; que doce!''

Conosco também é sempre assim: quando ouvimos o vento suspirando e
gemendo e choramingando em volta de nossas casas em noites solitárias,
não pensamos que nossos Anjos podem estar se lamentando por nossas
maldades, mas somente que uma tempestade está por vir. Os Anjos choravam
o tempo todo, e sentiam a tristeza da mudez -- pois, apesar de poder
falar, aqueles a quem falassem não os escutariam.

Enquanto o povo ria diante da ideia de Gigantes, havia um velho que
balançava a cabeça e lhes replicava, quando os ouvia, dizendo:

``A Morte tem muitos filhos, e ainda há Gigantes nos pântanos. Vocês
podem não vê"-los, talvez --, mas eles estão lá, e o único reduto de
segurança está numa terra de corações pacientes e leais.''

O nome desse bom velho era Knoal, e ele vivia numa casa construída com
grandes blocos de pedra, no meio de um local selvagem, longe da cidade.

Na cidade havia muitas casas velhas e grandes, com vários andares; e
nessas casas viviam muitas pessoas pobres. Quanto mais alto você subia
as grandes escadas íngremes, mais pobre era a gente que ali vivia, de
forma que nos sótãos havia pessoas tão pobres que, quando a manhã vinha,
não sabiam se teriam algo para comer durante o dia todo. Isso era
muito, muito triste, e crianças boas chorariam se vissem a sua dor.

Num desses sótãos vivia solitária uma mocinha chamada Zaya. Ela era uma
órfã, pois seu pai havia falecido há muitos anos e sua pobre mãe, que
havia trabalhado exaustivamente por muito tempo para sua querida
filhinha -- a única criança que tivera --, também havia morrido não
fazia muito.

A pobrezinha Zaya chorou tão amargamente quando viu sua querida mãe
morta, e ficou tão triste e desconsolada por tanto tempo, que se
esqueceu de que não tinha meios para viver. Entretanto, as pessoas
pobres que viviam na casa lhe davam parte de sua própria comida para que
ela não morresse de fome.

Então, depois de um tempo, ela tentou trabalhar por si mesma e ganhar o
próprio sustento. Sua mãe havia lhe ensinado a fazer flores de papel;
então fez um monte de flores e, quando juntou uma cesta cheia delas,
saiu para as ruas e as vendeu. Ela fazia flores de vários tipos, rosas e
lírios, violetas, fura"-neves, prímulas, resedas e muitas outras flores
belas que só crescem no País Sob o Pôr do Sol. Algumas dessas flores ela conseguia
fazer sem qualquer modelo, mas outras não; assim, quando queria um
modelo, tomava seu maço de folhas, tesouras, cola, pincéis e todas as
coisas que usava e ia para o jardim que pertencia a uma boa senhora onde
cresciam muitas flores belas. Ali se sentava e se punha a trabalhar,
observando as flores que queria.

Algumas vezes ficava muito triste, e suas lágrimas caíam espessas e
rápidas quando pensava em sua querida e falecida mãe. Muitas vezes,
parecia sentir que sua mãe a estava observando e vendo o terno sorriso
dela refletido na água à luz do sol; então seu coração se alegrava, e
ela cantava tão docemente que os pássaros a rodeavam e interrompiam seus
próprios cantos para escutá"-la.

Ela e os pássaros se tornaram grandes amigos, e às vezes, depois de
haver cantado uma música, eles todos, sentando"-se ao redor dela, entoavam
notas que pareciam dizer com bastante clareza:

``Cante para nós de novo. Cante para nós de novo.''

Então ela cantava de novo. Em seguida, pedia que eles cantassem, e eles
cantavam até que houvesse uma espécie de concerto. Após certo tempo, os
pássaros a conheciam tão bem que entravam em seu quarto e chegavam a
fazer ali mesmo seus ninhos e a segui"-la para onde quer que fosse. As
pessoas costumavam dizer:

``Olhe a menina dos pássaros. Ela mesma deve ser meio pássaro, pois
veja como os pássaros a conhecem e a amam.'' Porque tantas pessoas vinham a
dizer coisas como essa, alguns indivíduos tolos realmente acreditavam
que ela era meio pássaro e balançavam a cabeça quando pessoas mais
sábias riam delas, dizendo:

``Ela deve ser mesmo! Ouçam"-na cantando: sua voz é ainda mais doce do
que a dos pássaros.''

Então lhe foi dado um apelido; e garotos levados a seguiam pelas ruas
chamando"-a por ele. E~o apelido era ``Passarona''. Mas Zaya não se
importava com a alcunha; e, embora os garotos levados toda hora a
chamassem assim, pretendendo lhe causar sofrimento, isso não a
desagradava; ao contrário, pois se rejubilava tanto com o amor e a
confiança de seus queridinhos de voz doce que gostava de ser confundida
com eles.

De fato, seria bom para os garotinhos e as garotinhas levados se fossem
tão bons e inofensivos quanto os passarinhos que trabalham o dia todo
para seus filhotinhos indefesos, construindo ninhos e trazendo comida, e
sentando"-se pacientemente a chocar seus pequenos ovinhos manchados.

Certa noite, Zaya estava sentada sozinha em seu sótão, muito triste e
desolada. Era uma noite de verão muito agradável, e ela estava sentada
na janela, pousando os olhos na cidade. Podia ver muitas ruas que
iam em direção à grande catedral, cujo pináculo se erguia ao céu muito
mais alto que a grande torre do palácio do rei. Quase não havia sopro de
vento, e a fumaça subia reta das chaminés, tornando"-se cada vez mais
rala até desaparecer completamente.

\imagemmedia{}{./img/28.png}

Zaya estava muito triste. Pela primeira vez em muitos dias, seus
pássaros estavam todos longe dela, e ela não sabia aonde eles tinham ido.
Era como se a houvessem abandonado; e se sentia tão sozinha,
a pobrezinha, que derramou lágrimas amargas. Estava pensando na história
que há muito tempo sua falecida mãe havia lhe contado, a história de
como o Príncipe Zaphir havia matado o Gigante, e imaginou como era o
príncipe, e pensou como as pessoas devem ter sido alegres no tempo em que Zaphir
e Bluebell eram rei e rainha. Então se perguntou se havia crianças
famintas naqueles tempos bons, e se, de fato, como as pessoas diziam,
não mais havia Gigantes. Então pensou e pensou enquanto continuou a
trabalhar em frente à janela aberta.

De repente, desviou o olhar de seu trabalho e fitou o outro lado da
cidade. Lá viu uma coisa terrível -- algo tão terrível que emitiu um
gritinho de medo e espanto, e debruçou"-se na janela, fazendo sombra aos
olhos com sua mão para ver mais claramente.

No céu, além da cidade, viu uma Forma imensa e sombria, com os braços
erguidos. Estava envolta num grande manto de névoas,
desvanecendo"-se no ar, de modo que a menina só conseguia ver o rosto e as
mãos macabras, espectrais.

A Forma era tão portentosa que a cidade abaixo dela parecia um brinquedo
de criança. Estava ainda longe da cidade.

O coração da garotinha pareceu ficar paralisado de medo quando pensou:
``Os Gigantes, então, não estão mortos. Esse é mais um deles''.

Desceu correndo as altas escadas e saiu para a rua. Ali
viu algumas pessoas e gritou para elas:

``Olhem! Olhem! O Gigante, o Gigante!'', e apontou em direção à Forma
que ela ainda via se movendo lentamente na direção da cidade.

As pessoas olharam para cima, mas não podiam ver coisa alguma; então
riram e disseram:

``Essa criança está louca.''

Então a pobrezinha da Zaya ficou mais assustada do que nunca, e correu pela
rua, ainda gritando:

``Olhem! Olhem! O Gigante, o Gigante!'' Mas ninguém lhe prestou atenção
e todos disseram: ``Essa criança está louca'', e continuaram com seus
afazeres.

Então, os garotos levados se aproximaram dela e berraram:

``A Passarona perdeu seus colegas. Agora está vendo um pássaro maior do
que ela no céu e o quer para si.'' E~ficaram a fazer trovas sobre ela,
cantando"-as enquanto dançavam em círculo.

Zaya fugiu deles; correu apressada pelo meio da cidade, e adentrou os
campos mais além, pois ainda via a grande Forma diante de si, no ar.

À medida que avançava, e aproximava"-se mais e mais do Gigante, ele se
tornava um pouco mais escuro. Ela só conseguia enxergar as nuvens, mas ainda
era visível a forma turva de um Gigante pairando no ar.

Uma névoa fria rodeou"-a quando o Gigante pareceu vir em sua direção.
Então, pensou em todas as pessoas pobres na cidade, e teve esperança de
que o Gigante as poupasse; ajoelhando"-se diante dele, ergueu suas mãos
em súplica e gritou:

``Oh, grande Gigante! Poupe"-as, poupe"-as!''

Mas o Gigante seguia em frente como se não a tivesse escutado. Ela
gritou ainda mais alto:

``Oh, grande Gigante! Poupe"-as, poupe"-as!'' E curvou a cabeça e chorou;
e o Gigante, apesar de bem lentamente, continuava a avançar para a
cidade.

Não longe, havia um velho parado em pé, à porta de uma pequena casa
construída com grandes pedras, mas a menina não o viu. No~rosto dele
havia um olhar de medo e espanto, e, ao ver a criança se ajoelhar e
erguer as mãos, aproximou"-se e escutou sua voz. Quando a ouviu dizer
``Oh, grande Gigante'', murmurou para si mesmo:

``Então é mesmo como eu temia. Há mais Gigantes, e realmente esse é
outro deles.'' Olhou para cima, mas nada viu, e murmurou novamente:

``Eu não vejo, mas essa criança consegue ver; e, no entanto, eu temia,
pois algo me dizia que havia perigo. Realmente, o conhecimento é mais
cego do que a inocência.''

A menina, ainda sem notar qualquer ser humano por perto, gritou
novamente, soltou um grande grito de aflição:

``Oh!, não, não, grande Gigante, não faça mal às pessoas. Se alguém deve
sofrer, que seja eu. Leve"-me. Estou disposta a morrer, mas poupe"-as.
Poupe"-as, grande Gigante; e faça comigo o que bem entender.'' Mas o
Gigante não prestou atenção.

E Knoal -- pois era ele o velho -- sentiu seus olhos se encherem de
lágrimas, e disse a si mesmo:

``Oh, que nobre criança! Como é corajosa, está disposta a se
sacrificar!'' E, aproximando"-se, colocou a mão na cabeça da menina.

Zaya, cuja cabeça estava novamente arqueada, assustou"-se e olhou em
torno quando sentiu o toque. Entretanto, quando viu que era Knoal,
consolou"-se, pois sabia como ele era sábio e bom, e sentiu que, se
alguma pessoa poderia ajudá"-la, essa pessoa seria ele. Achegou"-se a ele e escondeu o
rosto em seu peito; ele fez carinho nos cabelos dela e a consolou. Mas,
ainda assim, não conseguia enxergar nada.

A névoa fria passou, e quando Zaya levantou os olhos, viu que o Gigante
já havia passado por ali e estava se movendo em direção a cidade.

``Venha comigo, minha filha'', disse o velho. E~os dois se levantaram e
entraram na casa construída com grandes pedras.

Quando Zaya entrou, ela espantou"-se, pois, pasmem!, o interior era como
uma tumba. O~velho percebeu seu arrepio, pois ainda a mantinha perto de
si, e disse:

``Não chore, pequenina, e não tema. Este lugar me lembra, e a todos que
nele entram, que à tumba todos retornaremos no fim. Não tema, pois este
se tornou um lar alegre para mim.''

Então a menina ficou aliviada, e começou a examinar mais atentamente seu
entorno. Viu todo tipo de instrumentos curiosos, e muitas ervas
estranhas e comuns, e plantas medicinais penduradas para secar em cachos
nas paredes. O~velho observou"-a em silêncio até que o medo dela
passasse, e depois disse:

``Minha filha, você viu a aparência do Gigante quando ele passou?''

Ela respondeu: ``Sim''.

``Pode descrever a face e o feitio dele para mim?'', perguntou
novamente.

Então ela começou a lhe contar tudo o que havia visto. Como o
Gigante era tão grande que todo o céu parecia preenchido. Como os
grandes braços estavam abertos, ocultos sob o manto, a ponto de, muito longe,
a mortalha se perder no ar. Como o rosto era o de um homem forte,
impiedoso, porém sem maldade; e como os olhos eram cegos.

O velho arrepiou"-se enquanto ouvia, pois percebeu que era um Gigante
muito terrível; e seu coração chorou pela malfadada cidade, onde tantos
haveriam de perecer em meio aos próprios pecados.

Eles decidiram partir e alertar novamente a malfadada população. Sem
atraso, o velho e a menina correram para a cidade.

Quando deixaram a casinha, Zaya viu o Gigante à frente deles, ainda
se movendo em direção a cidade. Apressaram"-se; e quando passaram por meio
da névoa fria, Zaya olhou para trás e viu que eles haviam ultrapassado o Gigante.

Rapidamente, chegaram à cidade.

Era uma visão estranha aquele velho e aquela menina correndo para avisar
as pessoas da terrível Praga que estava por cair sobre elas. A~longa
barba branca do velho e os cachos dourados da criança eram puxados para trás pelo vento,
de tão rápido que corriam. Os rostos de ambos estavam pálidos
como a morte. Atrás deles, visto apenas pelos olhos da mocinha de
coração puro quando olhava para trás, o espectral Gigante
continuava a avançar lentamente, toldando uma sombra escura no ar do
fim da tarde.

Mas as pessoas na cidade não viam o Gigante de modo algum. E~mesmo quando o
velho e a menina as alertavam, elas ainda assim não prestavam atenção,
mas zombavam e escarneciam deles, dizendo:

``Ora! Agora não há mais Gigantes'', e continuavam com seus afazeres,
rindo e zombando.

Então, o velho se colocou num lugar elevado entre eles, no degrau mais
baixo da grande fonte, com a menina ao seu lado, e falou assim:

``Oh!, povo, moradores deste Reino, sejam alertados a tempo. Esta
criança, de coração puro, em torno de cuja inocência até mesmo os
passarinhos, que temem os homens, e as mulheres reúnem"-se em paz, viu
esta noite no céu a forma de um Gigante que avança continuamente,
ameaçador, em direção à nossa cidade. Acreditem, oh!, acreditem; e
fiquem alertas enquanto podem. A~mim mesmo, como a vocês, o céu está
limpo; e, no entanto, vejam que eu acredito. Pois, escutem"-me: ignorando
completamente que um novo Gigante havia invadido nossa terra, sentei
pensativo em minha morada. E, sem causa ou motivo, veio ao meu coração
um medo repentino pela segurança de nossa cidade. Eu me levantei, olhei
ao norte e ao sul e ao leste e ao oeste, e para o alto e para baixo, mas
nunca pude enxergar algum sinal de perigo. Então eu disse a mim mesmo:
`Meus olhos estão turvos devido a uma centena de anos observando e
esperando, e assim não consigo enxergar.' E, no entanto, oh!, povo,
moradores deste reino, apesar de esse século ter embaçado meus olhos
externos, ele aguçou meus olhos internos -- os olhos de minha alma.
Novamente eu saí da minha casa e, veja!, esta menina estava ajoelhada,
implorando a um Gigante, invisível para mim, que poupasse a cidade; mas
ele não a escutou, ou, se escutou, não respondeu, e ela se prostrou no
chão. Então viemos para cá para alertá"-los. Dali, diz a menina, ele
avança para a cidade. Oh, sejam alertados! Alertados a tempo.''

Ainda assim, as pessoas não prestaram atenção, mas zombaram e riram
ainda mais, dizendo:

``Ora!, a menina e o velho estão loucos''; e foram para suas casas --
para dançar e festejar como antes.

Então os garotos levados vieram e zombaram deles, e~disseram que Zaya
perdera seus pássaros e ficara louca; e fizeram músicas e cantaram"-nas
enquanto dançavam em círculo.

Zaya estava tão dolorosamente angustiada pelo pobre povo que não prestou
atenção aos garotos cruéis. Vendo que ela não prestava atenção, alguns
se tornaram ainda mais rudes e malvados. Afastaram"-se um pouco e
arremessaram coisas contra a menina e o velho, zombando ainda mais dos
dois.

Então, triste no coração, o velho se levantou, e tomou a menina pela
mão, e a levou para bem dentro da floresta, abrigando"-a consigo na casa
feita de grandes pedras. Naquela noite Zaya dormiu com o doce cheiro de
ervas secas em torno de si, e o velho segurou sua mão para que não
tivesse medo.

De manhã, Zaya se levantou cedo e acordou o velho, que havia dormido em
sua cadeira.

Ela foi até a porta e olhou para fora; então uma vibração de alegria
sobreveio a seu coração. Pois, do lado de fora, como se esperando para
vê"-la, estavam todos os seus passarinhos e muitos, muitos mais. Quando
os pássaros viram a menina, entoaram alto alguns sons alegres, e voaram
loucamente ao redor de tanta alegria -- alguns deles sacudindo as
asas e tão engraçados que ela não conseguiu segurar algumas risadas.

Depois que Knoal e Zaya tomaram seu café da manhã simples e
repartiram"-no com seus amiguinhos de penas, partiram com corações
pesarosos para visitar a cidade e tentar mais uma vez alertar o povo. Os
pássaros voavam em torno deles enquanto andavam, e, para
incentivá"-los, cantavam o mais alegre que podiam, embora seus
coraçõezinhos estivessem abatidos.

Enquanto andavam, viam diante si o grande Gigante sombrio. E~agora ele
havia avançado até as fronteiras da cidade.

Mais uma vez alertaram as pessoas, e grandes aglomerados de gente os
cercaram, mas só para zombar deles mais do que nunca. E~garotos levados
jogaram gravetos e pedras nos passarinhos, matando alguns deles. A~pobre
Zaya chorou amargamente, e o coração de Knoal ficou muito triste. Depois
de certo tempo, quando os passarinhos já tinham se afastado da fonte em
que estavam, Zaya olhou para cima e teve um sobressalto com uma alegre
surpresa, pois não via mais o Gigante. Exclamou de alegria e as
pessoas riram, dizendo:

``Criança esperta! Ela vê que não vamos acreditar nela e agora finge que o
Gigante foi embora.''

Cercaram"-na, zombeteiros, e alguns deles disseram:

``Vamos colocá"-la na fonte e afundá"-la, como uma lição a mentirosos que
nos assustam.'' Eles então se aproximaram dela com ameaças. Ela se
agarrou a Knoal, que mostrava um semblante terrivelmente grave desde que~ela dissera que não via mais o Gigante. Ele estava como que em sonho,
pensativo. Mas, ao toque dela, pareceu acordar; e falou severamente às
pessoas, censurando"-as. Mas elas também gritaram com ele, e disseram
que, como havia ajudado Zaya em sua mentira, ele também seria afundado;
e se aproximaram para deitar as mãos em ambos.

A mão de um homem, que era um dos líderes do bando, já estava estendida,
quando emitiu um pequeno grito e apalpou a lateral do
próprio corpo; e, enquanto os outros se viravam para olhá"-lo,
assustados, gritou, sentindo uma grande dor e urrando horrivelmente.
Bem na hora que as pessoas o olharam, seu rosto começou a enegrecer e
enegrecer, e ele caiu diante delas, contorcendo"-se de dor por alguns momentos, e
então morreu.

Todas as pessoas gritaram de terror, e fugiram aos berros:

``O Gigante! O Gigante! Ele está mesmo entre nós!''

Temeram ainda mais por não poderem vê"-lo.

Mas antes de conseguirem sair da praça central, no meio da qual
estava a fonte, muitos caíram mortos, e os cadáveres ficaram no chão.

Ali no centro, o velho e a menina ajoelharam, rezando; e os pássaros
pousaram em volta da fonte, mudos e quietos, e nenhum som se escutava,
exceto os gritos das pessoas ao longe. Então, suas lamentações soaram
cada vez mais altas, pois o Gigante -- a Praga -- estava entre e em
torno deles, e não havia escapatória, pois agora era tarde demais para
fugir.

Ah! No País Sob o Pôr do Sol houve muito choro naquele dia. E~quando a
noite chegou, pouco se dormiu, pois havia medo em uns corações e dor
em outros. Ninguém estava quieto, exceto os mortos, que jaziam rígidos
espalhados pela cidade, tão inertes e sem vida que nem mesmo a fria luz da lua e as
sombras das nuvens passando sobre eles podiam fazer com que parecessem
vivos.

E por muitos dias houve dor e pesar e morte no País Sob o Pôr do Sol.

Knoal e Zaya fizeram tudo o que puderam para ajudar o pobre povo, mas
era realmente difícil ajudá"-los, pois o Gigante invisível estava entre
eles, vagando de lá e para cá pela cidade, de modo que ninguém podia
dizer sobre quem ele deitaria sua mão gelada da próxima vez.

Algumas pessoas fugiram da cidade; mas não adiantava, pois, como quer
que partissem dali e por mais rápido que fugissem, permaneciam ao
alcance do Gigante invisível. De vez em quando, com seu sopro e seu
toque, ele fazia de seus tépidos corações um gelo, e eles caíam mortos.

Alguns, como aqueles ficaram na cidade, eram poupados, mas uma parte
deles perecia de fome; já os demais rastejavam tristemente de volta para
a cidade, e viviam ou morriam entre seus amigos. E~tudo isso era, oh!,
tão triste, pois nada havia senão pesar e medo e choro de manhã à noite.

Agora, veja como os passarinhos amigos de Zaya ajudaram"-na em seu momento de
necessidade.

Eles aparentemente viram a vinda do Gigante quando ninguém -- nem mesmo a
menina -- pôde ver qualquer coisa, e conseguiam contar para ela quando
havia perigo, como se pudessem falar.

No começo, Knoal e ela iam para a casa feita de grandes pedras todas as
noites para dormir, e voltavam a cidade de manhã, ficando junto com o pobre
povo doente, consolando"-o e alimentando"-o, dando"-lhe os remédios que
Knoal, com sua grande sabedoria, sabia que lhes fariam bem. Assim,
salvaram muitas vidas humanas preciosas, e aqueles que foram salvos
ficaram muito agradecidos e, desde então e para sempre, viveram de
maneira mais pura e altruísta.

Depois de alguns dias, entretanto, descobriram que o pobre povo
doente precisava de ajuda mais de noite do que de dia, e então os dois
vieram para a cidade e nela passaram a morar juntos, ajudando o povo
abatido dia e noite.

De manhã bem cedo, Zaya saía para respirar o ar da manhã; e ali,
recém"-acordada do sono, estavam seus amigos emplumados esperando por
ela. Entoavam canções de alegria, achegavam"-se e empoleiravam"-se em seus
ombros e em sua cabeça, beijando"-a. Então, se ela fosse em direção a
qualquer lugar onde, durante a noite, a Praga houvesse deitado sua mão
mortífera, eles sempre se agitavam em frente a ela, tentavam impedi"-la e
gritavam em sua própria língua:

``Volte! Volte!''

Eles ciscavam de seu pão e bebiam de sua xícara antes que ela os
tocasse; e quando havia perigo -- pois a mão fria do Gigante estava por
toda parte --, sempre bradavam:

``Não, não!'', e ela não tocaria a comida, ou não deixaria qualquer um
tocá"-la. Frequentemente ocorria que, no exato momento em que ciscava o
pão ou bebia da xícara, um pobre passarinho caía, sacudia suas asas e
morria. Mas todos aqueles que morriam, morriam com um trilo de alegria,
olhando para sua pequena mestra, por quem haviam alegremente
falecido. Sempre que passarinhos achavam que o pão e a xícara estavam
puros e livres do perigo, olhavam para Zaya vivamente, e batiam suas
asas e tentavam piar, parecendo tão espertos que a pobre e triste
menininha sorria toda vez.

Havia um pássaro velho que sempre se demorava mais, e frequentemente
dava muito mais ciscadas no pão quando este era bom, de modo que
conseguia uma refeição substanciosa. E~algumas vezes continuava a se
alimentar até que Zaya balançava o dedo e lhe dizia:

``Guloso!'', e ele saltitava para longe como se não tivesse feito nada.

Havia outro passarinho querido -- um tordo, com peito tão vermelho
quanto o pôr do sol -- que amava Zaya mais do que se pode imaginar.
Quando experimentava a comida e descobria que era seguro comê"-la,
tirava~um pequeno pedacinho com o bico, voava e colocava na boca dela.

Cada passarinho que bebia da xícara de Zaya e gostava levantava sua
cabeça para agradecer; e desde então os passarinhos fazem a mesma coisa,
e nunca se esquecem de agradecer -- como o fazem algumas crianças
ingratas.

Assim viviam Knoal e Zaya, ainda que muitos à sua volta morressem, e o
Gigante ainda permanecia na cidade. Morriam tantas pessoas que
surpreendia o fato de que ainda sobrasse muita gente, pois foi só quando
a cidade começou a ficar rarefeita que o povo pensou no imenso número de
pessoas que havia vivido nela.

A pobre e pequenina Zaya ficara tão pálida e magra que parecia uma
sombra, e a figura de Knoal ficara mais curvada com os sofrimentos de
algumas poucas semanas do que estivera em todo seu centenário. Mas,
apesar de estarem fatigados e desgastados, os dois continuavam sua boa
obra de ajuda aos doentes.

Muitos dos passarinhos estavam mortos.

Uma manhã, o velho ficou muito fraco -- tão fraco que mal podia ficar em
pé. Zaya temeu por ele e disse:

``Você está doente, pai?'', pois agora ela sempre o chamava de pai.

Ele lhe respondeu com uma voz, ai!, rouca e baixa, mas muito, muito
carinhosa:

``Minha criança, temo que o fim esteja se aproximando: leve"-me para
casa, para que eu possa morrer lá.''

Às suas palavras, Zaya emitiu um pequeno grito e caiu de joelhos ao lado
dele, enterrando sua cabeça em seu peito, e chorou amargamente enquanto
o abraçava forte. Mas tinha pouco tempo para chorar, pois o velho
lutava para ficar em pé. Ao ver que ele desejava ajuda, ela limpou as
lágrimas e o ajudou.

O velho pegou seu cajado e, com Zaya ajudando a lhe dar apoio, chegou
até a fonte no meio da praça do mercado. E~ali, no degrau mais baixo,
sucumbiu, como que exausto. Zaya sentiu"-o ficar frio como gelo, e
percebeu que a mão gelada do Gigante havia se deitado sobre ele.

Então, sem saber o motivo, olhou para onde havia visto pela última vez o
Gigante, quando Knoal e ela estiveram ao lado da fonte. E~eis que, quando
olhou para lá, segurando a mão de Knoal, viu entre as nuvens, de
forma cada vez mais nítida, a forma sombria do terrível Gigante, o qual
estivera invisível por tanto tempo.

Seu rosto estava severo como sempre, e seus olhos ainda estavam cegos.

Zaya gritou ao Gigante, ainda segurando bem forte as mãos de Knoal:

``Ele não, ele não! Ó, poderoso Gigante! Ele não! Ele não!'', e arqueou
a cabeça e chorou.

Tamanha era a angústia em seu coração que dos olhos cegos do Gigante
sombrio brotaram lágrimas, que caíram como orvalho na testa do
velho. Knoal falou a Zaya:

``Não se aflija, minha menina. Estou contente por você enxergar novamente o
Gigante, pois tenho esperança de que ele irá deixar nossa cidade sem
infelicidades. Sou a última vítima, e morro contente.''

Então Zaya ajoelhou"-se em direção ao Gigante, e disse:

``Poupe"-o! Oh! Poupe"-o e me leve! Sim, poupe"-o! Poupe"-o!''

O velho soergueu"-se com os cotovelos, ainda deitado, e disse a ela:

``Não se aflija, pequenina, e não lamente. De verdade, sei que você
alegremente daria sua vida pela minha. Mas~nós devemos dar pelo bem dos
outros aquilo que para nós é mais caro do que nossas próprias vidas. A~bênção, minha pequenina, e seja boa. Adeus! Adeus!''

Quando pronunciou a última palavra, ficou frio como a morte, e seu
espírito partiu.

Zaya ajoelhou"-se e rezou; e quando olhou para cima, viu o Gigante
sombrio se afastando.

O Gigante virou"-se quando passou por ela, e Zaya viu que seus olhos
cegos estavam apontados em sua direção, como se ele tentasse enxergá"-la.
Ele levantou os grandes braços umbrosos, dobrou"-se em silêncio em sua
mortalha de névoa, como que a abençoando; e ela pensou que o vento que
passou por ela uivando carregava o eco das palavras:

``Inocência e devoção salvam o reino.''

\smallskip
Imediatamente viu ao longe o grande Gigante"-Praga se afastando para
as fronteiras do Reino, passando entre os Espíritos Guardiães e pelo
Portal em direção aos desertos mais além\ldots{} para sempre.
