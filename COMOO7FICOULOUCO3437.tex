\chapterspecial{Como o 7 ficou louco}{}{}
 

Na ribanceira do rio que corre através do Reino há um belo palácio no
qual mora um dos grandes homens.

A ribanceira se ergue íngreme da corredeira; e as grandes árvores
crescendo no sopé se erguem tão altas que seus galhos balançam no mesmo
nível das torres do palácio. É~um lugar belo, onde a grama é fresca e
curta e densa como veludo e verde como esmeralda. Ali, as margaridas
brilham como estrelas que caíram e jazem esparramadas pelo gramado.

Muitas crianças viveram e se tornaram homens e mulheres no velho
palácio, e eles tiveram muitos animais de estimação. Entre seus animais
havia muitos pássaros -- pois os pássaros, entre todas as espécies, amam
o lugar. Em um canto há um local que é chamado de Terra do Enterro dos
Pássaros. Aqui todos os animais são colocados quando morrem; e a grama
cresce mais viva aqui, e muitas flores brotam entre os monumentos.

Um dos garotos que aqui moraram teve, uma vez, como animal de estimação,
um corvo. Ele encontrou o pássaro, cuja pata havia sido machucada,
levou"-o para casa e cuidou dele até que ficou bom novamente; mas o
pobrezinho ficou manco.

Tineboy era o nome do jovem, e o pássaro se chamava Sr.~Gralha. Como
você pode imaginar, o corvo amava o garoto e nunca o deixou. Havia uma
gaiola para ele em seu quarto, e o pássaro ia todas as noites para ali
se empoleirar quando o sol se punha. Pássaros vão para cama regularmente
segundo seu entender; e, se você quisesse punir um pássaro, você o
acordaria. Pássaros não são como garotos e garotas. Imagine só punir um
garoto ou uma garota não os deixando ir para cama ao pôr do sol, ou
impedindo que eles se levantem bem cedo de manhã.

Bem, quando vinha a manhã, esse pássaro acordava e se alongava, piscava
os olhos e dava uma boa chacoalhada; então, sentia"-se muito acordado e
pronto para começar o dia.

É muito mais fácil acordar um pássaro do que um garoto ou uma garota. Não dá para cair sabão em seus olhos, ou o pente não irá se prender em nós
de cabelo, e seus cadarços nunca dão nós errados. Isso é porque ele não
usa sabão, ou pentes, ou cadarços; se usasse, talvez também ele
sofreria.

Quando o Sr.~Gralha acabava de se trocar, subia e tentava acordar
seu dono e fazê"-lo se levantar; mas, das duas, acordá"-lo era a tarefa
mais fácil. Quando o garoto ia para a escola, o pássaro voava na rua
junto a ele e se sentava em uma árvore próxima até que as aulas
acabassem; então, seguia"-o de volta para casa da mesma maneira.

\imagemmedia{}{./img/12.png}


Tineboy gostava muito do Sr.~Gralha e algumas vezes tentava fazê"-lo
entrar em sua sala durante as aulas. Mas o pássaro era muito sábio e não
entrava.

Um dia, Tineboy estava às voltas com seus problemas de matemática e, ao
invés de prestar atenção ao que estava fazendo, ficou tentando fazer o
Sr.~Gralha entrar. O~problema era ``multiplicar 117.649 por 7''. Tineboy
e o Sr.~Gralha ficaram olhando um para o outro. Tineboy fez sinais para
o pássaro entrar. O~Sr.~Gralha, entretanto, não se moveu; ele se sentou
na sombra no lado de fora, pois o dia estava muito quente, pendeu sua
cabeça para um lado e observou sabiamente.

``Entre, Sr.~Gralha'', disse Tineboy,``e me ajude a resolver esse
problema''. O~Sr.~Gralha somente grasnou.

``Sete vezes nove são setenta e sete, sete vezes nove são setenta e
nove… não, noventa e sete. Oh, eu não sei… queria que o
número sete nunca tivesse sido inventado'', disse Tineboy.

``Grá'', disse o Sr.~Gralha.

O dia estava muito quente e Tineboy estava muito sonolento. Ele pensou
que talvez seria capaz de resolver melhor o problema se ele descansasse
um pouquinho, só para pensar; e assim, abaixou sua cabeça na carteira.
Ele não estava muito confortável, pois sua testa estava em cima 7, ao menos
ele achou que estava; assim, ele a mudou de posição até que ficou bem na
beirada da carteira. Então, depois de um tempo, de alguma forma, coisas
muito estranhas começaram a acontecer.

O Professor estava prestes a contá"-los uma história.

Todos os alunos haviam se acomodado para escutar; o Corvo se sentou no
peitoril da janela aberta, pendeu sua cabeça para um lado, fechou um
olho -- o olho mais perto da sala de aula -- para que pensassem que
estava dormindo, e escutou mais atentamente do que qualquer um deles.

\imagemmedia{}{./img/26.png}

Os pupilos estavam todos felizes -- todos, exceto três. Um porque sua
perna dormiu; outra porque ela tinha o bolso cheio de coalhada e queria
comê"-la, mas não conseguiria comer sem ser descoberta, e a coalhada
estava derretendo; e o terceiro estava com muito sono e muito ansioso
para ouvir a história, mas não podia fazer uma coisa por causa da outra.

O mestre, então, começou sua história.

\emph{\versal{COMO} O \versal{POBRE} 7 \versal{FICOU} \versal{LOUCO}}


\emph{O~Médico de Alfabeto…}

Aqui ele foi interrompido por Tineboy, que perguntou:

``O que é um Médico de Alfabeto?''

``Um Médico de Alfabeto'', disse o mestre, ``é o médico que cuida das
doenças e das enfermidades das letras do Alfabeto''.

``Como Alfabetos têm doenças e enfermidades?'', perguntou Tineboy.

``Oh, eles têm muitas. Você nunca fez um `o' torto ou um `A' maiúsculo
com uma perninha manca, ou um `T' que não tivesse as costas retas?''

Houve um coro de toda a sala: ``Ele faz. Ele faz bastante''. Ruffin, o
garoto maior, disse após todos os outros: ``Bastante mesmo. Na verdade,
sempre''.

``Muito bem, então deve haver alguém para deixá"-las retas novamente,
não?''

Nenhuma das crianças pôde dizer que não. Ouviu"-se Tineboy, sozinho,
murmurar para si mesmo: ``\emph{Eu não acredito}''\emph{.}

O mestre recomeçou…

\emph{O~Médico do Alfabeto estava sentado tomando chá. Ele estava muito
cansado, pois esteve cuidando de casos o dia todo.}

Tineboy interrompeu de novo: ``Quais casos?''

``Posso lhe contar. Ele teve de colocar um `i' que havia sido omitido, e
alterar a perna de um `R' que havia se tornado um `B'\,''.

\emph{Bem, logo quando ele estava começando a tomar seu chá, houve uma
batida rápida na porta. Ele foi até ela, abriu"-a, e um estribeiro entrou
apressado, sem fôlego por causa da corrida, e disse:}

\emph{``Oh, Médico, venha rápido! Há uma calamidade em nossa casa.}

\emph{``Qual é a sua casa?'', perguntou o médico.}

\emph{``Oh, você sabe. Os Estábulos dos Números.''}

``O que são os Estábulos dos Números?'', perguntou Tineboy,
interrompendo novamente.

``Os Estábulos dos Números'', disse o Professor, ``são os estábulos onde
os números são guardados''.

``Por que eles são guardados em estábulos?'', perguntou Tineboy.

``Porque eles vão muito rápido.''

``Como eles vão muito rápido?''

``Pegue um problema, resolva"-o e você verá imediatamente. Ou olhe na sua
tabuada: começa com duas vezes um são dois e antes que você chegue ao
fim da página você estará em doze vezes doze. Isso não é ir rápido?

``Bem, eles têm de guardar os números em estábulos, senão todos os
números iriam fugir e nunca mais se ouviria falar deles. No fim do dia
todos eles voltam para casa, trocam os sapatos, limpam"-se e jantam.''

\emph{O~Estribeiro dos Estábulos dos Números estava muito impaciente.}

\emph{``Oh, pobre 7, senhor.''}

\emph{``O que houve com ele?''}

\emph{``Ele está quase morrendo. Pensamos que ele nunca irá
conseguir.''}

\emph{``Conseguir o quê?'', perguntou o Médico.}

\emph{``Venha ver'', disse o Estribeiro.}

\emph{O~Médico apressou"-se, levando a lanterna consigo, pois a noite
estava escura, e logo chegou aos Estábulos.}

\emph{Quando ele se aproximou, escutou"-se um som muito curioso -- um som
ofegante e engasgado, gemidos e tosse, risadas, e um berro selvagem e
sobrenatural, tudo ao mesmo tempo.}

\emph{``Oh! Venha rápido!'', exclamou o Estribeiro.}

\emph{Quando o Médico entrou nos estábulos lá estava o pobre número 7
com todos os vizinhos em volta dele, e ele estava muito mal. Estava
espumando pela boca e aparentemente louco. A~Enfermeira da Vila da
Gramática estava segurando"-o pela mão, tentando sangrá"-lo. Todos os
vizinhos estavam apertando com força as mãos ou os pescoços, ou estavam
ajudando a segurá"-lo. O~Pezeiro, o homem}, explicou o professor, vendo
pela expressão no rosto de Tineboy que ele iria fazer uma pergunta, ``o
homem que coloca os pés nas letras e nos números para que eles fiquem em
pé sem se cansar'' \emph{-- estava segurando o pobre número louco.}

\emph{A~Enfermeira, tentando acalmá"-lo, disse:}

\emph{``Pronto, pronto, querido… não faça barulho. Chegou aqui o
bom Médico de Alfabeto, que vai deixar você são.''}

\emph{``Não vou ficar são'', disse o 7, bem alto.}

\emph{``Mas, meu caro senhor'', disse o Médico,``isso não pode
continuar. Certamente você não está louco o bastante para insistir em
estar louco?''}

\emph{``Sim, estou'', disse o 7, bem alto.}

\emph{``Então'', disse o Médico suavemente, ``se você está louco o
bastante para insistir em estar louco, devemos tentar curar sua loucura
ou o seu estar louco, e então você ficará lúcido o bastante para querer
não estar louco, e curaremos isso também.''}

``Eu não estou entendendo'', disse Tineboy.

``Shh!'', fez a classe.

\emph{O~Médico tomou seu estetoscópio, seu telescópio, seu microscópio e
seu horóscopo e começou a utilizá"-los no pobre e louco 7.}

\emph{Primeiro ele colocou o estetoscópio na sola de seu pé e começou a
falar nele.}

\emph{``Não é assim que se usa isso,'' disse a Enfermeira. ``Você deve
colocá"-lo no peito dele e depois auscultar.''}

\emph{``De maneira alguma, minha cara senhora,'' disse mansamente o
Médico, ``esse é o jeito que se faz nas pessoas sãs; mas, claro, quando
alguém está louco, o caso da doença precisa de um método oposto de
tratamento.'' Então, ele tomou o telescópio e observou para verificar o
quão perto ele estava, e o microscópio para ver o quanto era pequeno.
Então ele sacou seu horóscopo.}

``Por que ele o sacou?'', perguntou Tineboy.

``Porque, meu filho querido'', disse o Professor, ``você não vê que por
direito um horóscopo é feito? Mas, porque o pobre homem estava louco, o
horóscopo havia de ser sacado''.

``O que é um horróscopo?'', perguntou Tineboy.

``Não é horróscopo, meu filho; é um horóscopo -- uma coisa muito
diferente''.

``Bem, o que é horóscopo?''

``Procure em seu dicionário, querido'', respondeu o Professor.

\emph{Bem, quando o médico terminou de usar todos os seus instrumentos,
ele disse: ``Uso tudo isso a fim de descobrir o alcance da doença.
Agora, começarei a encontrar a causa. No primeiro momento, interrogarei
o paciente''.}

\emph{``Bem, meu bom senhor, por que você insiste em estar louco?''}

\emph{``Por que assim escolho.''}

\emph{``Oh, meu caro senhor, essa não é uma resposta polida. Por que
você escolhe?''}

\emph{``Não posso dizer o porquê'', disse o 7, ``a menos que eu faça um
discurso.''}

\emph{``Bem, faça um discurso.''}

\emph{``Não posso falar até que eu seja posto em liberdade; como posso
fazer um discurso com todas essas pessoas me segurando?''}

\emph{``Estamos com medo de te soltar,'' disse a Enfermeira, ``você irá
fugir''.}

\emph{``Não vou''.}

\emph{``Você promete?'', perguntou o médico.}

\emph{``Eu prometo'', disse o 7.}

\emph{``Soltem"-no'', disse o Médico e, dessa forma, eles colocaram um
pedaço de tapete sob ele; o Pezeiro sentou"-se em sua cabeça, da maneira
que fazem quando cavalos caem na rua. Depois, todos se distanciaram, e o
Pezeiro também se distanciou. Depois de uma longa luta, o 7 se
levantou.}

\emph{``Agora, faça o discurso'', disse o Médico.}

\emph{``Não posso começar'', disse o 7, ``até que eu tenha um copo de
água em uma mesa. Quem já ouviu de qualquer um discursando sem um copo
de água?!''}

\emph{Então eles trouxeram um copo de água.}

\emph{``Senhoras e Senhores…'', iniciou o 7, e então parou.}

\emph{``O que está esperando?'', perguntou o Médico.}

\emph{``Por um aplauso, claro'', disse o 7. ``Quem já ouviu falar de um
discurso sem aplausos?''}

\emph{Todos eles aplaudiram.}

\emph{``Estou louco'', disse o 7, ``porque eu escolho estar louco; e
nunca irei, serei, poderei, deverei, seria, poderia ou viria a ser
qualquer coisa além de louco. O~tratamento que recebo é o suficiente
para me deixar louco''.}

\emph{``Ora, ora!'', disse o Médico. ``Que tratamento?''}

\emph{``De manhã, à tarde e à noite sou tratado pior do que qualquer
escravo. Não há, em todo o alcance do aprendizado, qualquer coisa que
tenha tanto a suportar quanto eu tenho. Trabalho duro o tempo todo.
Nunca resmungo. Sou frequentemente um múltiplo, frequentemente um
multiplicando. Estou disposto a suportar meu quinhão de ser um
resultado, mas não posso aguentar o tratamento que recebo; e, além
disso, eles não são órfãos como eu''.}

\emph{``Órfãos?'', perguntou o Médico, ``o que quer dizer?''}

\emph{``Quero dizer que os outros números têm muitas relações. Mas não
tenho parentes nem família -- exceto o velho Número 1, e ele não conta
muito; e, além disso, sou somente seu ta"-ta"-ta"-ta"-taraneto''.}

\emph{``De que maneira?'', perguntou o Médico.}

\emph{``Oh, ele é um velho camarada que está presente o tempo todo. Ele
tem todos os seus filhos à sua volta, e eu venho somente seis gerações
depois''.}

\emph{``Hunf!'', exclamou o Médico.}

\emph{``O Número 2'', continuou o 7, ``nunca entra em confusão, e o 4, o
6 e o 8 são seus primos. O~Número 3 é próximo do 6 e do 9. O~Número 5 é
um meio décimo e nunca se mete em confusão. Mas, quanto a mim, sou um
miserável, maltratado e sozinho''. Aqui o pobre 7 começou a chorar e,
arqueando sua cabeça, soluçou amargamente.}

Quando o Professor chegou a esse ponto houve uma interrupção, pois aqui
o pequeno Tineboy também começou a chorar.

``Por que está chorando?'', perguntou Ruffin, o garoto brigão.

``Não estou chorando'', disse Tineboy, e soluçou mais rápido do que
nunca.

O Professor continuou a história.

\emph{O~Médico de Alfabeto tentou alegrar o pobre 7.}

\emph{``Escute, escute!'', disse ele.}

\emph{O~7 parou de chorar e olhou para ele. ``Não'', ele disse, ``você
deve dizer `fale, fale'; sou eu quem deveria dizer `escute,
escute'\,''.}

\emph{``Certamente'', disse o Médico, ``você diria isso se fosse são;
mas, por outro lado, você não é são, e, estando louco, você diz o que
não deveria dizer''.}

\emph{``Isso é falso'', disse o 7.}

\emph{``Eu entendo'', disse o Médico, ``mas não interrompa para discutir
esse ponto. Se você fosse são você diria `isso é verdade', mas você diz
`isso é falso', querendo dizer que concorda comigo''.}

\emph{O~7 pareceu satisfeito em ser tão compreendido.}

\emph{``Não'', disse ele -- querendo dizer ``sim''.}

\emph{``Então'', continuou o Médico, ``se você disser `fale, fale',
quando um homem são diria `escute, escute', claro, eu diria `escute,
escute' quando quisesse dizer `fale, fale' porque estou falando com um
louco''.}

\emph{``Não, não'', disse o 7 -- querendo dizer ``sim, sim''.}

\emph{``Continue seu discurso'', disse o Médico.}

\emph{O~Número 7 pegou seu lenço e chorou.}

\emph{``Senhoras e senhores'', ele continuou, ``mais uma vez eu devo
advogar a causa do número pobre e mal"-usado -- que sou eu -- este número
órfão -- este número sem parentes…''}

Aqui Tineboy interrompeu o Professor: ``Como ele não tinha aparentes?''

``Parentes, minha criança. Parentes, e não aparentes'', disse o
Professor.

``Qual a diferença entre parentes e aparentes?'', perguntou Tineboy.

``Ficará muito pouco aparente'', disse o Professor, ``a diferença entre
esta bengala e seu couro se você interromper''. Assim, Tineboy ficou
quieto.

``\emph{Bem}'', seguiu o professor, ``\emph{o pobre 7 continuou: imploro
sua piedade para este número miserável. Oh, vocês, garotos e garotas,
pensem em um pobre número desolado, que não tem casa, nem amigos, nem
pai, mãe, irmão, irmã, tio, tia, sobrinho, sobrinha, filho, filha ou
primo, e está desolado e sozinho}''.

Aqui, Tineboy soltou um urro terrível.

``Por que está chorando?'', perguntou o Professor.

``Eu quero que o velho e pobre 7 seja mais feliz. Eu darei a ele um
pedaço de meu lanche e uma parte da minha cama''.

O Professor voltou"-se ao Monitor.

``Tineboy é uma boa criança'', ele disse, ``faço"-o, para a próxima
semana, aprender 7 vezes 0 em diante e talvez isso irá reconfortá"-lo''.

O Corvo, sentado na janela, piscou seu olho para si mesmo e saltitou no
entorno com um grasnado contido e contente, balançou suas asas, e
pareceu estar se abraçando e rindo. Então, saltitou para longe, subiu
com a ponta das patas e se escondeu em cima da estante de livros.

O Mestre continuou sua história.

\emph{Bem, crianças, depois de um tempo o pobre 7 melhorou e prometeu
que ele ficaria deslouco. Antes de o Médico ir novamente para casa,
todas as Crianças Alfabeto e Número vieram e apertaram a mão do pobre
Número 7, e prometeram que eles seriam mais bonzinhos com ele no
futuro.}

\smallskip
``Então, crianças, o que vocês acham da história?''

Todos elas disseram que gostaram, que era bela, e que tentariam também
ser mais bonzinhos com o pobre 7 no futuro. Por fim, Ruffin, o valentão,
disse:

``Eu não acredito. E, se for verdade, eu gostaria que ele tivesse
morrido; ficaríamos melhor sem ele''.

``Ficaríamos?'', perguntou o professor. ``Como?''

``Porque não nos importaríamos com ele'', disse Ruffin.

No momento em que ele disse isso, ouviu"-se um tipo de grasnado esquisito
emitido pelo Corvo, mas ninguém se importou, exceto Tineboy, que disse:

``\emph{Sr.~Gralha, você e eu amamos o pobre 7, em todo caso}''.

O Corvo odiou Ruffin porque ele sempre jogava pedras nele, e tentara
puxar as penas de sua cauda; e enquanto Ruffin falava, seu grasnado
parecia significar: ``Espere só''. Quando ninguém mais estava olhando, o
Sr.~Gralha saltitou para cima e se escondeu nas vigas.

Então, na mesma hora, a escola acabou e Tineboy foi para casa. Mas ele
não conseguiu achar o Sr.~Gralha. Pensou que ele estivesse perdido, que
estava muito infeliz, e foi para cama chorando.

Nesse meio tempo, quando a escola estava trancada e vazia, o Sr.~Gralha
saiu das vigas muito, muito quietamente -- saltitou por sobre a porta e,
abaixando sua cabeça, escutou; então, voou e escalou a maçaneta da
porta, e olhou pela fechadura. Não havia nada para ver e nada para
ouvir.

Então, ele se ergueu na mesa do Mestre, bateu suas asas, e começou a
grasnar como um galo, porém muito suavemente, com medo de ser ouvido.

Imediatamente, sobrevoou toda a sala, voando até as grandes folhas da
tabela de multiplicação, virando as folhas dos livros com suas garras e
pegando Alguma Coisa com seu bico afiado.

Seria difícil de acreditar, mas ele estava roubando todos os Números
Sete daquele lugar; retirou o Sete do relógio, raspou"-o da lousa e
borrou"-o do quadro negro com suas asas.

O Sr.~Gralha sabia que, uma vez que você tirasse a inteireza de qualquer
número de uma escola, ninguém mais poderia usá"-lo sem pedir sua licença.

Enquanto ele estava tirando todos os Setes, inchou muito; e quando ele
os retirou todos, ficou exatamente Sete vezes maior do que seu
tamanho natural.

Ele não foi capaz de fazer tudo isso de uma vez. Levou"-lhe a noite toda,
e quando voltou para seu canto nas vigas, era quase hora de a escola
abrir.

Ele estava agora tão grande que conseguiu somente se espremer no canto e
mais nada.

A hora da escola chegou, mas não havia Mestre e não havia alunos. Toda
uma hora passou; e então o Mestre chegou, e os Bedéis, e todos os
garotos e as garotas.

Quando todos eles estavam na sala, o Mestre disse:

``Vocês todos estão muito atrasados''.

``Por favor, senhor, não pudemos evitar''.

Eles todos responderam juntos…

``Por que não puderam evitar?''

``Não fui acordado a tempo''.

``A que horas vocês são acordados toda manhã?''

Todos eles pareceram prestes a falar, mas ficaram calados.

``Por que não respondem?'', perguntou o Professor.

Eles fizeram movimentos com suas bocas como se estivessem falando, mas ninguém
disse nada.

O Corvo, em seu canto, emitiu um grasnado, rindo silenciosamente só para
si mesmo.

``Por que não respondem?'', perguntou novamente o Professor. ``Se não
responderem imediatamente à minha pergunta, vou mantê"-los todos aqui
dentro''.

``Por favor, senhor, não conseguimos'', disse um aluno.

``Por que não?''

``Porque…''

Aqui, Tineboy interrompeu, ``\emph{Por que se atrasou tanto, senhor?}''.

``Bem, meu filho, peço desculpas por ter me atrasado; mas o fato é que
meu criado não bateu à minha porta na hora normal''.

``\emph{Que hora, senhor?}'', perguntou Tineboy.

O Professor pareceu como se fosse falar, mas parou.

``Isso é muito estranho'', ele disse, depois de uma longa pausa.

Ruffin disse, de uma maneira fanfarrona: ``Não estamos atrasados de modo
algum. Você está aqui e nós estamos aqui -- isso é tudo''.

``Não, isso não é tudo'', disse o Professor. ``As horas são dez, e agora
são onze -- perdemos uma hora''.

``Como perdemos uma hora?'', perguntou um dos alunos.

``Bem, isso é o que está me confundindo. Precisamos esperar um pouco
para ver''.

Aqui, Tineboy disse de repente: ``\emph{Talvez alguém roubou!}''

``Roubou o quê?'', perguntaram os alunos.

``\emph{Não sei}'', disse Tineboy.

Todos riram.

``\emph{Vocês não precisam rir, algo foi roubado; olhem para a minha
lição!}'', disse Tineboy e segurou alto o livro. Aqui está o que eles
viram:

--

1

são

--

--

2

``

14

--

3

``

21

--

4

``

28

--

5

``

35

--

6

``

42

--

--

``

49

--

8

``

56

--

9

``

63

--

10

``

--0

Todos os alunos rodearam Tineboy para olhar o livro. Ruffin não foi,
pois ele estava olhando o relógio da escola.

``O relógio perdeu alguma coisa'', ele disse, e com certeza não parecia
certo.

O Professor olhou para cima -- pois ele estava inclinado com sua cabeça
em sua mesa, grunhindo.

``O que há de errado?'', ele perguntou.

``Algo está faltando''.

``Falta um número; há somente onze números'', disse o Professor.

``Não, não'', disseram os alunos.

``Conte"-os, Ruffin'', pediu o Mestre.

``1 2 3 4 5 6 8 9 10 11 12''.

``Certo'', disse o Professor, ``você vê que há doze números. Não, não há
-- sim, há -- não -- sim -- não, sim -- o que está havendo?'', e olhou
em torno da sala, e inclinou sua cabeça novamente em sua mesa e grunhiu.

Nesse meio tempo, o Corvo havia rastejado pelas vigas até que chegou
acima da mesa do Professor; e então ele pegou um Sete grande e pesado e
deixou"-o cair no pequeno pedaço careca no topo da cabeça do Professor.
Ele rebateu na cabeça e caiu na mesa em frente a ele. No instante em que
o Professor o viu, descobriu o que estava querendo o tempo todo. Ele
cobriu o Sete com um pedaço de papel borrão. Então, chamou Ruffin.

``Ruffin, você me disse que algo estava faltando -- tem certeza?''

``Sim, claro''.

``Muito bem. Lembra"-se que você disse ontem que queria que um certo
Número morresse em um manicômio?''

``Sim, eu lembro; e ainda quero''.

``Bem, esse Número foi roubado por alguém durante a noite''.

``Viva!'', disse Ruffin e jogou seu livro ao teto. Ele acertou o pobre
Sr.~Corvo, que tinha outro Sete em seu bico prestes a deixá"-lo cair, e
deixou cair esse Sete. Ele caiu no chapéu de Tineboy, que o segurou sem
sua mão. Pegou"-o, inclinou"-se e fez"-lhe carinhos.

``\emph{Pobre 7}'', disse Tineboy.

``Me dê o Número'', disse Ruffin.

``\emph{Não darei. Ele pertence a mim}''.

``Então vou te obrigar'', disse Ruffin. Ele agarrou Tineboy -- mesmo na
frente do rosto do Mestre.

``\emph{Me deixa. Não te darei meu pobre Sete}'', disse Tineboy, e ele
começou a gritar e a chorar.

``Ruffin, afaste"-se'', ordenou o Mestre.

Ruffin afastou"-se.

``Sete vezes sete?'', perguntou o Mestre.

Ruffin não respondeu. Ele não o poderia, pois ele não tinha um Sete.

``\emph{Eu sei}'', disse Tineboy.

``Ah, sim'', disse Ruffin, com um olhar de desprezo, ``ele sabe porque
tem um Número''.

``\emph{Quarenta e nove}'', disse Tineboy.

``Correto'', disse o Mestre; ``venha para a frente, Tineboy''.

Então Tineboy foi para a frente da sala, e Ruffin para trás.

``Sete vezes quarenta e nove?'', perguntou o Mestre.

Todos ficaram em silêncio.

``Vamos, respondam!'', exclamou o Mestre.

``\emph{Quanto é? Sim, você mesmo!}'', disse Tineboy.

``Bem, meu filho, perdão, mas não posso falar. Céus, é muito estranho'',
e o Mestre abaixou sua cabeça em sua mesa novamente, e grunhiu mais alto
do que nunca.

Bem nesse momento, o Sr.~Gralha tomou outro Sete e derrubou"-o no chão na
frente de Tineboy.

``Trezentos e quarenta e três,'' disse Tineboy, rapidamente; pois agora
ele podia responder, já que tinha outro Sete.

O Professor levantou a cabeça e riu alto.

``Viva! Viva!'', ele disse.

Quando o terceiro Sete caiu, o Corvo começou a inchar.

Ele ficou sete vezes maior do que era, de forma que começou a levantar
as telhas do telhado.

Todos os alunos olharam para cima; Ruffin tinha sua boca aberta, e o Sr.~Gralha, ansioso por se livrar dos Setes, soltou um dentro dela.

``Dois mil, quatrocentos e um'', Ruffin balbuciou.

O Sr.~Gralha soltou outro Sete em sua boca, e ele balbuciou novamente,
mais do que nunca: ``Dezesseis mil, oitocentos e sete''.

O Corvo começou a atirar Setes nele tão rápido quanto podia; e a cada
vez que ele atirava um Sete, ficava menor e menor, até que ficou do seu
tamanho natural.

Ruffin continuou a balbuciar e a ofegar números tão rápido quanto jamais
pôde, até que o rosto ficou preto e ele caiu em convulsão assim que
chegou a ``setenta e nove mil e setecentos e noventa e dois bilhões,
duzentos e sessenta e seis mil e duzentos e noventa e sete milhões,
seiscentos e doze mil e um''.

De repente, Tineboy acordou e viu que estivera sonhando com sua cabeça
abaixada.

\imagemmedia{}{./img/13.png}
