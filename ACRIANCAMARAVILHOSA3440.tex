\chapterspecial{A criança maravilhosa}{}{}
 
\ \


Muito longe, na beira de um grande riacho, que se estendia para o
distante interior do mar infinito, havia uma tranquila vila.

Ali os lavradores levavam uma vida feliz e próspera. Eles se levantavam
cedo, de forma que na manhã fria e gris eles ouviam a cotovia,
totalmente invisível nas alturas da manhã, cantando o hino matinal do
qual ela nunca se esquece.

Quando o pôr do sol vinha furtivo, eles retornavam a suas casas, felizes
pelo que o resto do cair da noite trazia a eles.

No outono, quando a colheita devia ser feita, eles trabalhavam até
tarde, conforme eram capazes de fazer; pois, naquela época, o bom Sol e
sua esposa, a Lua, tinham um pacto de que ajudariam aqueles que
trabalhassem na colheita. Então o sol ficava no céu um pouco mais, e a
lua saía de sua cama no horizonte um pouco mais cedo; assim, havia
sempre luz para se trabalhar.

A lua rubra, larga e cheia, que olha de cima aos lavradores trabalhando,
é chamada de Lua da Colheita.

O Senhor da Mansão dessa vila tranquila era um homem muito bom e
agradável, que ajudava sempre os pobres. Na hora da refeição, a porta de
sua mansão ficava aberta e todos os que estavam com fome poderiam entrar
se assim escolhessem e tomar assento na mesa, sendo hóspedes bem"-vindos.

Esse Senhor da Mansão tinha três filhos, Sibold e May, e um Garotinho
havia acabado de chegar em casa, ainda sem ter um nome.

Sibold havia acabado de alcançar seu décimo oitavo aniversário e May
estava a dois meses de seu sexto. Eles gostavam muito um do outro --
como irmãos e irmãs devem gostar -- e faziam todas as brincadeiras
juntos. May pensava que Sibold era muito grande e forte, e qualquer
coisa que ele desejasse fazer ela sempre concordava.

Sibold adorava achar coisas e fazer explorações; e em tempos diferentes
as duas crianças estiveram por todo os domínios de seu pai.

Eles tinham certos abrigos secretos dos quais ninguém sabia exceto eles
mesmos. Alguns deles eram lugares extraordinários e deleitáveis.

Um ficava no centro de um carvalho oco, no qual viviam tantos esquilos
que os galhos eram quase como ruas de uma cidade devido às idas e vindas
deles.

Outro lugar era no topo de uma rocha, que era alcançado somente por um
caminho estreito entre arbustos altos de heras. Aqui havia um tipo de
cadeira grande cinzelada na terra, em que cabia somente os dois; e para
ali eles frequentemente levavam seu almoço e sentavam"-se pela metade do
dia observando sobre os topos das árvores onde, bem longe na distância,
a borda alva do horizonte deitava"-se no mar cintilante.

Então eles contavam um ao outro as coisas que pensavam e o que gostariam
de fazer e o que tentariam fazer quando crescessem.

Havia também outro lugar que lhes era o favorito entre todos.

Era embaixo de um grande Salgueiro Chorão. Essa era uma árvore vigorosa,
com muitas centenas de anos, que se erguia alta acima de outras árvores
que pontuavam a relva. Os longos galhos caíam tão espessos que, até
mesmo no inverno, quando as folhas tinham caído e os galhos estavam nus,
mal se podia ver no buraco oco que havia ali dentro.

Quando a nova roupagem da primavera voltava a toda a árvore, de seu cimo
mais alto até o solo musgoso do qual ela se elevava, ocorria uma
abundância de verde sólido; e era difícil entrar nela mesmo se se
soubesse o caminho.

Em um lugar, um dos galhos longos tinha sido, há muito tempo, quebrado
em uma grande tempestade que havia posto abaixo muitas árvores da
floresta. Mas os galhos que pendiam perto dele lançaram novos brotos
para preencher o espaço vazio, e, assim, a abertura foi coberta com
ramos finos ao invés de galhos fortes.

No verão, as folhas cobriam tudo com uma massa de verde; mas aqueles que
conheciam a abertura podiam empurrar de lado os ramos e, assim, entrar
no desvão.

Era um desvão muito belo. Sem importar o quão forte o sol brilhasse lá
fora, dentro era fresco e agradável. Desde o chão até mesmo o topo, até
o próprio telhado em que os galhos pretos que se encontravam formavam
uma massa escura, tudo era de um verde delicado, pois a luz lá de fora
vinha pelo meio das folhas suave e docemente.

Sibold e May pensavam que assim o mar devia parecer às Sereias, que
cantam e penteiam seus longos cabelos com pentes dourados nas
profundezas frias do oceano.

Na relva ao redor dessa grande árvore havia muitos canteiros de belas
flores. Ásteres, com seus rostos largos de muitas cores, olhando fixos
diretamente ao sol sem mesmo piscar os olhos, ao redor e por onde
esvoaçavam belas borboletas, com suas asas como arco"-íris ou pavões ou
pores do sol ou nada que fosse mais belo. A~doce Reseda, sobre a qual
pairavam abelhas com zumbido agradecido. Amores"-perfeitos, com seus
rostos grandes e delicados tremendo em seus caules delgados. Tulipas,
abrindo suas bocas ao sol e à chuva; pois a Tulipa é uma flor
gananciosa, que abre tanto sua boca até que, de tão aberta, sua cabeça
quebra"-se em pedaços e ela morre. Jacintos, com seus muitos sinos
agrupados em um galho -- como uma grande festa de família. Grandes
Girassóis, cujos rostos pendentes brilhavam como filhos do próprio
parente, o Sol.

Havia também grandes Papoulas, com folhas espraiadas e descuidadas,
caules grossos e suculentos, e grandes flores escarlates, que se erguiam
e pendiam como bem quisessem, e que pareciam muito livres e descuidadas
e independentes.

Tanto Sibold quanto May amavam essas Papoulas e iam todos os dias
olhá"-las. Nos canteiros, na relva musgosa, da qual se erguia o grande
Salgueiro, elas cresciam a tamanhos enormes; tão altas que, quando
Sibold e May ficavam de mãos dadas junto ao canteiro, as grandes
Papoulas elevavam"-se sobre eles, até que Sibold, ficando na ponta dos
pés, não pudesse alcançar as flores escarlates.

Um dia, depois do café da manhã, Sibold e May levaram consigo sua comida
e saíram para passar o dia juntos passeando entre os bosques, pois era
uma festa para eles. Um pequenino irmão Garotinho havia chegado na casa
e todos estavam ocupados arrumando coisas para ele. As crianças
haviam"-no visto somente um instante.

De mãos dadas, Sibold e May percorreram todos os seus lugares favoritos.
Eles olharam a caverna no Carvalho e diziam ``Como vai o senhor?'' a
todo esquilo que vivia na árvore, e contaram"-lhes sobre o novo Bebê que
havia chegado em casa. Depois eles foram à rocha, e sentaram"-se juntos
no assento, e olharam o mar distante.

Eles ficaram ali por um tempo sob a luz do sol quente e falaram do
pequeno e querido irmãozinho bebê que haviam visto. Eles se perguntaram
de onde ele havia vindo, e fizeram um plano; procurariam e procurariam
até que eles também encontrassem um bebê. Sibold disse que ele deve ter
vindo lá do mar e ter sido colocado no canteiro de salsa pelos Anjos
para que uma enfermeira pudesse encontrá"-lo ali e levá"-lo para confortar
sua pobre mãe doente. Depois eles pensaram como seriam capazes de partir
para além do mar, e planejaram que algum dia o barco de Sibold seria
aumentado e eles entrariam nele e partiriam pelos mares, e procurariam
outro bebezinho todo para eles.

Depois de um tempo eles se cansaram de se sentar no sol quente; então,
deixaram o lugar e, de mãos dadas, perambularam até que chegaram na
relva plana onde o grande Salgueiro estava, e onde os canteiros de
flores faziam o ar parecer cheio de cor e de perfume.

De mãos dadas eles caminharam, olhando as borboletas, e as abelhas, e os
pássaros, e as belas flores.

Em um canteiro encontraram uma nova flor que aparecera. Sibold
conhecia"-a e contou a May que era um Lírio Asiático; ela teve medo de se
aproximar dele até que ele contou a ela que a flor não a machucaria,
pois era somente uma flor.

À medida que caminhavam, Sibold colhia algumas flores de cada canteiro e
dava"-as à sua irmã; quando eles estavam se afastando do Lírio Asiático,
ele puxou a flor, e, porque May tinha medo de carregá"-la, ele mesmo a
levou.

Por fim, eles chegaram ao grande canteiro das Papoulas. As flores
pareciam tão brilhantes e frescas, devido a toda a sua cor, e tão
descuidadas, que May e Sibold pensaram, ambos ao mesmo tempo, que elas
gostariam de tomar parte com eles no Desvão do Salgueiro, pois eles
estavam indo comer lá e desejaram que o lugar estivesse tão feliz e belo
quanto possível.

Mas, antes, eles voltaram ao Carvalho para recolher muitas folhas, pois
Sibold sugeriu que fariam do novo bebê irmão o Rei do Banquete e que
eles iriam fazer a ele uma coroa de carvalho. Como ele não estaria lá em
pessoa, eles colocariam a coroa onde eles a pudessem ver bem.

Quando chegaram ao Carvalho, May exclamou:

``Oh, olhe, Sibold, olhe, olhe!''

Sibold olhou e viu que em quase todos os galhos havia um monte de
esquilos sentados dois a dois, com suas caudas cheias de pelos sobre
suas costas, comendo nozes tão ávidos quanto podiam.

Quando os esquilos os viram, não tiveram medo, pois as crianças nunca
haviam feito mal algum a eles. Eles emitiram todos juntos um tipo de
grasnido estranho e um pequeno pulo engraçado. Sibold e May começaram a
rir, mas eles não gostaram da perturbação e voltaram ao canteiro de
Papoulas.

``Agora, Sibold, querido'', disse May, ``precisamos pegar muitas
Papoulas, pois o querido Be gosta muito delas''.

``Como você sabe?'', perguntou Sibold.

``Porque ele deve gostar'', ela respondeu. ``Você e eu gostamos, e ele é
nosso irmão, então claro que ele gosta''.

Então Sibold colheu muitas Papoulas e algumas delas ele apanhou com
muitas das folhas verdes presas até que ficaram, cada um, com um braço
cheio delas. Então juntaram todas as outras flores e entraram no Desvão
do Salgueiro para comer. Sibold foi à fonte que nascia no jardim e que
corria ao mar. Ali ele encheu seu gorro com água e trouxe"-o tão firme
quanto pôde para que não derramasse muito e voltou ao desvão. May
segurou abertos os galhos folhados quando ele chegou; quando passou, ela
os deixou cair novamente. Quando a cortina de folhas estava pendurada
toda em volta deles, as duas crianças ficaram sozinhas no Desvão do
Salgueiro.

Então se puseram a trabalhar para adornar sua cabana folhada com as
flores. Eles as torciam em torno dos galhos que pendiam, e fizeram uma
coroa de flores que eles colocaram em volta do tronco da árvore. Em toda
parte, eles colocaram as Papoulas no lugar mais alto que alcançavam, e
então Sibold segurou May no alto enquanto ela enfiava os Lírios
Asiáticos em uma fissura no tronco de árvore em cima de todas as outras
flores.

Então as crianças sentaram"-se para comer. Os dois estavam muito cansados
e com muita fome, e apreciaram muito o descanso e a comida. Havia
somente uma coisa que eles queriam, e essa coisa era o novo Irmãozinho
Bebê, para que pudessem fazê"-lo o rei do banquete.

Quando a refeição terminou, eles se sentiram muito cansados, então se
deitaram juntos com suas cabeças uma no ombro do outro e seus braços
entrelaçados; e ali eles foram dormir com as Papoulas escarlates
acenando em todo o entorno.
\ \


Depois de um tempo, eles não estavam mais dormindo. Não parecia ser mais
tarde no dia, mas sim ser de manhã cedo. Nenhum deles se sentiu nem um
pouco sonolento ou cansado; ao contrário, ambos queriam partir para uma
expedição mais longa do que nunca.

``Venha ao riacho'', disse Sibold, ``e saiamos em meu barco''.

May levantou"-se, e eles abriram a porta de folhas e saíram. Desceram ao
riacho e lá encontraram o barco de Sibold com todas as suas velas
armadas.

``Vamos entrar'', disse Sibold.

``Por quê?'', perguntou May.

``Porque assim podemos velejar'', ele respondeu.

``Mas ele não vai nos aguentar; é pequeno demais'', disse May, que
estava com um tanto de medo de velejar, mas não queria dizê"-lo.

``Tentemos'', disse seu irmão. Ele tomou a corda que amarrava o barco à
margem e puxou. A~linha parecia muito longa, e Sibold parecia estar
puxando"-a já por um bom tempo. Entretanto, o barco por fim chegou. À~medida que se aproximava, ele ficava cada vez maior, até que, quando
tocou a margem, eles viram que era grande o bastante para aguentar os
dois.

``Vamos, vamos entrar'', disse Sibold.

De alguma forma May não sentia mais medo. Ela entrou no barco e
descobriu que ali havia almofadas de seda da cor das flores de Papoula.
Então Sibold entrou e afastou a corda que amarrava o barco à margem. Ele
sentou"-se na popa e segurou o leme em sua mão; May sentou"-se em uma
almofada no fundo do barco e segurou"-se nas bordas.

As velas brancas inflaram com uma brisa suave e eles começaram a se
afastar da margem; as pequenas ondas agitaram"-se desde a proa do barco.
May ouvia o marulhar das ondas quando elas tocavam a proa, e então se
deitou.

O sol brilhava muito vivamente. A água estava tão azul quanto o céu e
tão límpida que as crianças podiam ver lá nas profundezas, onde os
peixes estavam se movendo rápidos. Ali, também, as plantas e as árvores
que crescem sob a água estavam abrindo e fechando seus galhos; e as
folhas estavam se movendo como o fazem aquelas das árvores terrestres
quando sopra o vento.

Por um momento, o barco afastou"-se da margem, até que eles perderam a
vista do alto Salgueiro que era maior do que os outros. Então pareceu se
aproximar novamente à margem, e mudou de posição, sempre tão perto que
as crianças podiam ver muito claramente tudo o que lá havia.

A margem era muito variada e cada momento mostrava algo novo e
belo…

Agora era uma rocha saliente toda coberta com plantas rastejantes cujas
flores quase tocavam a água.

Agora era uma praia, em que a areia branca reluzia e resplandecia à luz,
e na qual as ondas faziam um zunido agradável à medida que corriam
margem acima e novamente voltavam -- como se brincando ``tocando''
consigo mesmas.

Agora árvores escuras com uma folhagem densa pendiam sobre a água; mas,
através de sua obscuridade, brilhavam fendas distantes quando o sol
jorrava, por alguma abertura, na clareira.

Novamente, havia lugares em que a grama, tão verde quanto esmeralda,
seguia em declive até a borda da água, e onde as Prímulas e os
Ranúnculos que cresciam na margem, sobre a qual se debruçavam, quase
beijavam as ondinhas que iam encontrá"-los.

Então havia lugares em que grandes Lilases tornavam o ar, até bem longe,
doce com o sopro de seus cachos de flores rosa e branca, e onde os
Laburnos pareciam jorrar torrentes de ouro da riqueza das flores que
pendiam de seus galhos verdes retorcidos.

Havia também grandes Palmeiras, com suas folhas amplas, fazendo uma
sombra fresca na terra abaixo. Grandes Coqueiros, em cujos troncos
tropas de macacos ficavam correndo para reunir cocos que eles arrancavam
e jogavam para baixo. Aloés com grandes caules carregados com flores
púrpuras e douradas -- pois esse era o centésimo ano quando, somente
então, os aloés florescem.

Havia Papoulas tão grandes quanto árvores e Lírios cujas flores eram
maiores que cabanas.

As crianças gostaram de todos esses lugares, mas, de repente, chegaram a
um lugar em que havia um canteiro de grama esmeralda ensombrado por
árvores gigantescas. Em volta crescia ou pendia ou se agrupava cada uma
das flores que crescem. Canas"-de"-Açúcar altas brotavam da beirada de um
pequeno córrego que fluía sobre um leito de pedras brilhantes como
joias. Cervos elevavam suas cabeças eminentes, e plantas com grandes
folhas erguiam"-se e produziam sombras até mesmo na penumbra. Perto havia
uma fonte cristalina que borbulhava formando pequeno córrego de onde as
Canas"-de"-Açúcar se erguiam.

Quando eles viram esse lugar, ambas as crianças bradaram: ``Oh! Que
bonito! Vamos parar aqui''.

O barco pareceu entender os desejos deles, pois sem o leme ser tocado,
virou"-se e fluiu suavemente à margem.

Sibold desceu e levantou May para a terra. Ele pretendia amarrar o
barco; mas, no instante que May saiu, todas as velas se dobraram por si
mesmas, a âncora pulou para fora do barco e, antes que fosse possível
fazer qualquer coisa, o barco estava ancorado perto da margem.

Sibold e May deram"-se as mãos e perambularam juntos pelo lugar, olhando
para tudo.

Logo, May disse, em um sussurro:

``Oh, Sibold, este lugar é tão bom, será que há Salsa aqui?''

``Por que você quer Salsa?'', ele perguntou.

``Porque, se houver um bom canteiro de Salsa, talvez poderemos encontrar
um Bebê… E, oh!, Sibold, quero \emph{muito} um Bebê''.

``Muito bem, então, vamos procurar'', disse seu irmão. ``Parece haver
todo tipo de planta aqui; e se há \emph{todo} tipo de planta, você sabe
que \emph{deve} haver Salsa''. Pois Sibold era muito lógico.

Então as duas crianças caminharam por todo o vale gramado, procurando.
Logo depois, decerto, sob as folhas espalhadas de uma Cidra, eles
encontraram um grande canteiro de Salsa -- as maiores Salsas que eles já
haviam visto.

Sibold ficou bem satisfeito com isso, e disse: ``Isso se parece com
Salsa. Sabe, May, sempre fiquei intrigado com como um Bebê, que é muito
maior do que a Salsa, possa estar escondido nela; e ele deve estar
escondido nela, pois muitas vezes saio para olhar no canteiro de casa e
nunca consigo achar um, apesar de a enfermeira sempre achar um em
qualquer lugar que ela vá olhar. Mas ela não procura quase nunca. Sei
que, se eu fosse tão sortudo quanto ela, ficaria sempre procurando''.

May viu que o desejar encontrar um bebê tornou"-se tão forte nela que
disse novamente:

``Oh, Sibold, eu desejo \emph{tanto} um Bebê; \emph{espero} que
encontremos um''.

Quando ela falou, ouviu"-se um som estranho -- um tipo de risada muito,
muito leve -- como um sorriso causado por música.

May ficou surpresa e, por um momento, não pensou em fazer coisa alguma;
ela meramente apontou, e disse:

``Olhe, olhe!''

Sibold correu adiante e levantou a folha de uma enorme Salsa; e ali --
oh, alegria de alegrias! -- estava deitado o Bebê mais precioso que já
fora visto.

May ajoelhou"-se ao lado dele, levantou"-o, começou a balançá"-lo e cantou
``Nana nenê'', enquanto Sibold olhava complacente. Entretanto, depois de
uns instantes, ele ficou impaciente e disse:

``Veja bem, entende, eu encontrei esse Bebê; ele pertence a mim''.

``Oh, por favor'', disse May, ``eu o ouvi primeiro. Ele é meu''.

``Ele é meu'', disse Sibold. ``Ele é meu'', disse May; e ambos começaram
a ficar um pouco nervosos.

De repente eles ouviram um gemido baixo -- um tipo de som como se uma
música tivesse dor de dente. Ambas as crianças olharam para baixo
alarmadas e viram que o pobre Bebê estava morto.

Ambos ficaram horrorizados e começaram a chora; e pediram perdão um ao
outro e prometeram que nunca, nunca mais iriam ficar nervosos. Quando o
fizeram, a Criança abriu seus olhos, olhou para eles gravemente, e
disse:

``Agora, nunca briguem ou fiquem nervosos. Se ficarem nervosos de novo,
qualquer um dos dois, eu morrerei, sim, e serei enterrado também, antes
que vocês possam dizer `raquetes'\,''.

``De fato, Be'', disse May, ``nunca, nunca ficarei brava de novo. Ao
menos, eu tentarei não ficar''.

Disse Sibold:

``Eu lhe garanto, senhor, que sob nenhuma provocação, resultando de
quaisquer concatenações de circunstâncias, eu serei culpado da
\emph{malfaisance} da raiva''.

``Como ele fala bonito'', disse May; e o Bebê acenou a ele com sua
cabeça de maneira familiar, como dizendo:

``Tudo bem, velho, nós nos entendemos''.

Então, por um tempo, todos eles ficaram cada vez mais quietos. Logo, o
Bebê virou seus olhos azuis para May e disse:

``Por favor, pequena mãe, cantaria para mim?''

``O que gostaria, Be?'', perguntou May.

``Oh, qualquer coisinha, algo patético'', ele respondeu.

``Algum estilo em particular?'', perguntou May.

``Não, obrigado; qualquer coisa que venha a calhar. Prefiro algo simples
-- alguma coisinha elementar, como, por exemplo, qualquer canção
começando com uma escala cromática em quintas e oitavas consecutivas,
\emph{pianíssimo} -- \emph{rallentando} -- \emph{excellerando} --
\emph{crescendo} -- até uma mudança harmônica na dominante da nona bemol
diminuta''.

``Oh, por favor, Be'', disse May, muito humildemente, ``não sei ainda
nada sobre isso. Estou ainda nas escalas e, com sua licença, não sei do
que tudo isso se trata''.

``Olhe, e você verá'', disse a Criança, e tomou um pedaço de graveto e
escreveu uma música na areia.

``Ainda não sei'', disse May.

Bem naquele momento, um animal marrom"-amarelado pequeno apareceu na
clareira caçando um rato. Quando ficou no lado oposto deles, de repente
disparou como o som de uma pistola.

``Agora você sabe?'', perguntou a Criança.

``Não, querido Be, mas não importa'', ela respondeu.

``Muito bem, querida'', disse a Criança, beijando"-a, ``qualquer coisa
que lhe agradar, somente deixe vir diretamente de seu coraçãozinho
amável'', e ele a beijou novamente.

Então May cantou algo muito doce e belo -- tão doce e belo que a fez
chorar, e também Sibold, e o Bebê. Ela não conhecia as palavras e ela
não conhecia a melodia, e ela tinha somente uma noção bem vaga do que
falava; mas era muito, muito bela. Durante todo o tempo enquanto ela
cantava, ela cuidou do Bebê e ele colocou seus bracinhos gordos em volta
do pescoço dela, e a amou muito.

Quando ela terminou de cantar, a Criança disse:

``Chlap, Chlap, Chlap, M"-chlap!''

``O que ele quer dizer?'', perguntou Sibold, desconfortado, pois ela viu
que o Bebê queria algo.

Naquele momento, uma bela Vaca colocou sua cabeça sobre os arbustos e
disse: ``Muu"-uu"-uu''. A~Bela Criança bateu suas palmas; assim também
May, que disse:

``Oh, eu sei agora. Ele quer ser alimentado''.

A Vaca entrou ali sem ser convidada, e Sibold disse:

``Acho que sim, May, melhor eu tirar leite dela''.

``Por favor, sim, querido'', disse May. Ela começou a ninar novamente o
Bebê, a beijá"-lo, a acalentá"-lo, e a contar"-lhe que logo iria ser
alimentado.

Enquanto ela estava assim ocupada, sentava"-se com suas costas a Sibold.
Mas o Bebê estava olhando para a ordenhação, com seus olhos azuis
dançando com alegria. Subitamente, ele começou a rir, rir tanto que May
olhou em volta para ver do que ele estava rindo. Ali estava Sibold
tentando ordenhar a Vaca puxando seu rabo.

A Vaca não parecia se importar com ele, e continuou a pastar.

``Eia, Dona'', disse Sibold. A~Vaca começou a mover"-se.

``Oh, nossa!'', disse Sibold, ``vamos, apresse"-se e dê"-nos um pouco de
leite; o Be quer um pouco''.

A Vaca respondeu a ele:

``O querido Be não deve desejar nada''.

May pensou que era muito estranho a Vaca poder falar; mas, como Sibold
não pareceu pensar isso ser estranho, segurou a língua.

Sibold começou a discutir com a Vaca: ``Mas, agora, Senhora Vaca, se ele
não deve querer nada, por que a senhora faz ele querer?''

A Vaca respondeu: ``Não me culpe. A~culpa é sua. Tente de outro jeito'',
e começou a rir tão alto quanto podia.

Sua risada era muito engraçada, a princípio muito alta, mas gradualmente
ficando mais e mais parecida com a risada da Criança, até que May não as
conseguia diferenciar. Então, a Vaca parou de rir, mas a Criança
continuou.

``Do que está rindo, Be?'', May perguntou, pois ela não lembrava se
sabia algo de ordenhação além do que Sibold sabia. Ela achou isso muito
engraçado, pois sabia que muitas vezes ela havia visto as vacas serem
ordenhadas em casa.

O Bebê falou: ``Não é assim que se ordenha uma vaca''.
\ \


Então Sibold começou a levantar e abaixar o rabo da Vaca como a haste de
uma bomba; mas o Bebê riu ainda mais.

Subitamente, sem saber como isso veio a acontecer, ela se sentiu
derramando leite de um regador em cima do bebê todo, que estava deitado
no chão, com Sibold segurando sua haste. O~Bebê estava vociferando e
rindo como louco; e quando o regador ficou vazio, ele disse:

``Muito obrigado aos dois. Nunca apreciei tanto um jantar em minha
vida''.

``Esse é um Be muito querido e estranho!'', disse May, em sussurros.

``Muito'', disse Sibold.

Enquanto falavam, veio um som terrível de entre as árvores, muito muito
longe a princípio, mas que se aproximava mais e mais a cada momento. Era
como gatos que estavam tentando imitar o trovão. O~barulho veio
bombardeando através das árvores.

``Meiau"-u-room"-r-p"-sss. Rarkrrau"-iau"-p-ss''.

May ficou muito assustada. Também Sibold ficou assustado, mas ele não
iria admitir; ele sentiu que tinha de proteger sua Irmãzinha e o Bebê,
então se pôs entre os dois e o lugar de onde vinha o som. May abraçou
apertado a Criança, e disse"-lhe: ``Não tenha medo, querido Be. Nós não
vamos deixar ele tocar em você''.

``O que é `ele'?'', perguntou o Bebê.

``Eu não sei, Be'', ela respondeu. ``Gostaria de saber. Lá vem ele
agora'', pois, exatamente naquele instante, um grande e nervoso Tigre
surgiu sobre os topos das árvores mais altas e ficou lá, olhando
furiosamente para eles com seus grandes olhos verdes flamejantes.

May olhou para essa coisa terrível com seus olhos arregalados de terror;
mas, ainda assim, ela abraçou o Bebê cada vez mais forte. Ficou olhando
para o Tigre e viu que ele estava mirando não ela nem Sibold, mas sim o
Bebê. Isso a fez mais assustada do que nunca e agarrou"-o ainda mais
apertado. Enquanto olhava, no entanto, ela percebeu que os olhos do
Tigre ficaram cada vez menos bravos a cada instante que passava, até
que, por fim, eles ficaram tão gentis e amansados quanto aqueles do seu
próprio gato malhado favorito.

Então o Tigre começou a rugir. O~rugido era como o rugido de um gato,
mas tão alto que parecia tambores. Entretanto, ela não se importou com
isso, pois, apesar de alto, parecia como se fosse gentil e carinhoso.
Então o Tigre se aproximou e agachou"-se diante da Criança Maravilhosa, e
lambeu suas mãozinhas gordas com sua grande e áspera língua vermelha,
porém muito suavemente. O~Bebê riu e acariciou o grande focinho do
Tigre, puxou os longos bigodes eriçados e disse:

``Gii, gii''.

O Tigre passou a se comportar de maneira muito engraçada. Ele se deitou
de costas e rolou ao redor, depois ficou em pé e rugiu mais alto do que
nunca. Sua grande cauda ergueu"-se diretamente para cima, com a ponta
movendo"-se ao redor e derrubando aqui e ali um monte de uvas que pendiam
da árvore acima. Parecia inundado de alegria, veio e agachou"-se
novamente ante a Criança, e rugiu em volta dele em grande estado de
alegria. Finalmente, deitou"-se, sorrindo e rugindo, e guardando a
Criança, como se de guarda.

Logo depois veio da distância outro som terrível. Era como um grande
Gigante sibilando; e era mais alto do que um trem, e mais numeroso do
que um bando de gansos. Havia também o som de galhos se partindo, de
esmagamento da vegetação rasteira, e havia um som terrível de algo sendo
carregado como nada que eles já ouviram antes.

Novamente Sibold se prostrou entre o som e May, que, mais uma vez,
segurou o Bebê para protegê"-lo do mal.

O Tigre levantou"-se e arqueou suas costas como um gato bravo e ficou
pronto para avançar em qualquer coisa que viesse.

Então ali apareceu, sobre os topos das árvores, a cabeça de uma enorme
Serpente, com olhos miúdos que brilhavam como fagulhas de fogo e duas
grandes mandíbulas abertas. Essas mandíbulas eram tão grandes que
realmente parecia como se toda a cabeça do animal estivesse aberta em
duas; e entre elas aparecia uma grande língua ramificada que parecia
cuspir veneno. Atrás dessa cabeça monstruosa apareceram do corpo da
Serpente enormes chocalhos que se moviam continuamente. O~Tigre rugiu
como se a ponto de saltar; mas, de repente, a Serpente baixou sua cabeça
em submissão. Estava fitando a Criança Maravilhosa; e, olhando, May
também viu que o pequenino Bebê estava apontando, como se dando ordens à
Serpente, a seus pés. Então o Tigre, com um rosnado baixo e depois um
rugido contente, voltou ao seu lugar para vigiar e ficar de guarda. A~grande Serpente veio suavemente e se enrolou na clareira, e também
parecia como se estivesse vigiando e de guarda para a Criança
Maravilhosa.

Novamente veio outro som terrível. Dessa vez, fora no ar. Grandes asas
pareciam bater com um som mais alto do que o trovão; e, de longe, o ar
foi escurecido por uma portentosa Ave de Rapina que lançava uma sombra
sobre a terra com suas asas abertas.

Quando a Ave de Rapina desceu, o Tigre levantou"-se novamente e arqueou
suas costas como se prestes a pular e avançar sobre ela, e a Serpente
levantou seus chocalhos poderosos e abriu suas mandíbulas como se
prestes a dar o bote.

Mas quando a Ave viu a Criança ela também se tornou menos feroz, e
suspendeu"-se no meio do ar com sua cabeça inclinada como se estivesse
reverenciando. Logo, a Serpente enrolou"-se como antes, o Tigre voltou a
vigiar e ficar de guarda, e a Ave de Rapina pousou na clareira e ficou
vigiando e também de guarda.

May e Sibold começaram a observar maravilhados o Belo Garoto, ante a
quem esses monstros faziam reverência; mas eles não conseguiam ver nada
de estranho.

Novamente houve outro som terrível -- dessa vez do mar -- uma arremetida
e um assovio como se alguma coisa gigante estivesse chicoteando a água.

Olhando em torno, as crianças viram dois monstros se aproximando. Eram
um Tubarão e um Crocodilo. Eles surgiram do mar e vieram para a terra. O~Tubarão estava pulando, batendo com sua cauda e rangendo sua tripla
fileira de grandes dentes. O~Crocodilo estava rastejando com seus
grandes pés e pernas curtas e curvas, e sua boca terrível estava abrindo
e fechando, batendo seus grandes dentes.

Quando esses dois se aproximaram, o Tigre e a Serpente e a Ave de Rapina
todos se ergueram para proteger a Criança; mas quando os recém"-chegados
viram o Bebê, eles também reverenciaram e também mantiveram vigia e
guarda -- o Crocodilo rastejando na praia, e o Tubarão movendo"-se para
lá e para cá na água -- como sentinelas.

Novamente May e Sibold olharam para a Bela Criança e espantaram"-se.

Mais uma vez houve um som terrível, mais horrível do que se havia
escutado.

A terra pareceu tremer e um som profundo e abafado veio de muito abaixo.
Então, um pouco longe, uma montanha ergueu"-se de repente; seu cume
abriu"-se, e dali estouraram, com um som mais alto do que o de uma
tempestade, fogo e fumaça. Grandes volumes de vapor preto levantaram"-se
e suspenderam"-se, uma nuvem preta, acima. Pedras fervendo de tamanho
enorme foram atiradas para o alto e caíram novamente na cratera, e
perderam"-se. Pelos lados da montanha rolavam torrentes de lava
incandescente e fontes de água fervorosa irrompiam de todo lado.

Sibold e May ficaram mais temerosos do que nunca e May agarrou o querido
Bebê forte contra seu peito.

O troar da montanha flamejante ficou cada vez mais alto, a lava ardente
jorrava densa e rápida, e da cratera ergueu"-se a cabeça de um feroz
Dragão, com olhos como carvão incandescente e dentes como línguas de
fogo.

Então o Tigre e a Serpente e a Ave de Rapina, e o Crocodilo e o Tubarão,
todos se preparam para defender a Criança Maravilhosa.

Mas quando o feroz Dragão viu o Garoto, ele, também, domou"-se e rastejou
humildemente para fora da cratera em chamas.

Então a montanha furiosa afundou novamente para dentro da terra, a lava
incandescente desapareceu; e o Dragão permaneceu com os outros para
vigiar e ficar de guarda.

Sibold e May ficaram mais impressionados do que nunca, e olharam para o
Bebê com ainda maior curiosidade. De repente, May disse a seu irmão:

``Sibold, quero cochichar a você algo''.

Sibold inclinou sua cabeça e ela sussurrou muito baixo em seu ouvido:

``Eu acho que Be é um Anjo!''

Sibold olhou para ele pasmo e respondeu:

``Eu também acho, querida. O~que vamos fazer?''

``Não sei'', disse May, ``espero que ele não fique bravo de novo conosco
por o chamarmos de `Be'\,''.

``Espero que não'', disse Sibold.

May pensou por um momento e então seu rosto iluminou"-se com um sorriso
contente quando ela disse:

``Ele não ficará bravo, Sibold. Você sabe que nós o divertimos sem
querer''.

``Bem verdade'', disse Sibold.

Enquanto estavam falando, todos os tipos de animais e pássaros e peixes
estavam vindo à clareira, andando de braços dados, tanto quanto podiam
-- pois nenhum deles tinha braços. Um Leão e um Carneiro vieram
primeiro, e estes dois curvaram"-se à Criança, e depois se foram e se
deitaram juntos. Então veio uma Raposa e um Ganso; e depois um Gavião e
um Pombo; e depois um Lobo e outro Cordeiro; e depois um Cachorro e um
Gato; e depois outro Gato e um Rato; e depois outra Raposa e uma
Cegonha; e uma Lebre e uma Tartaruga; e um Lúcio e uma Truta; e um
Pardal e uma Minhoca; e muitos, muitos outros, até que toda a clareira
estivesse cheia de coisas vivas, todas em paz uma com a outra.

Todos eles sentaram"-se em volta da clareira em pares e todos eles
olharam para a Criança Maravilhosa.

May sussurrou novamente para Sibold:

``Acho que, se ele for um anjo, devemos ser muito respeitosos com ele''.

Sibold assentiu, mostrando que concordava com ela; então ela aconchegou
o Bebê mais perto e disse:

``Por favor, senhor Be, eles não parecem todos bons e belos, sentados
assim?''

A Bela Criança sorriu docemente quando respondeu:

``Belos e doces eles parecem''.

May disse novamente:

``Gostaria que eles sempre fossem assim, e nunca brigassem ou
discordassem de forma alguma, querido Be. Oh! Peço perdão. Digo, Senhor
Be''.

A Criança perguntou a ela:

``Por que pede meu perdão?''

``Porque lhe chamei de Be, ao invés de Senhor Be''.

O Garoto perguntou novamente:

``Por que você deveria me chamar de Senhor Be?''

May não gostaria de dizer ``Porque você é um Anjo'' da forma como
gostaria de ter dito, então ela aproximou mais a Criança e sussurrou em
seu pequeno ouvido róseo:

``Você sabe''.

A criança colocou seus bracinhos em volta do pescoço dela e beijou"-a, e
disse, bem baixo e bem docemente, palavras que por toda sua vida ela não
se esqueceu:

``Eu sei. Seja sempre carinhosa e doce, querida criança, e até mesmo os
Anjos conhecerão teus pensamentos e escutarão tuas palavras''.

May sentiu"-se muito feliz. Olhou para Sibold, que se inclinou e
beijou"-a, e chamou"-a de ``doce irmãzinha''; e todos os animais, em
pares, e todos aqueles terríveis que estavam de guarda, todos disseram
juntos, como uma aclamação:

``Certo!''

Então eles pararam e emitiram todos juntos cada um dos sons, um após o
outro, que cada um usava para mostrar que estava feliz. Primeiro todos
rugiram, e depois todos grasnaram, e depois todos cacarejaram, e
grunhiram, e bateram as asas e sacudiram suas caudas.

``Oh, que bonito!'', disse May novamente, ``olhe, querido Be!'' Ela
estava prestes a dizer Senhor quando a Criança levantou seu dedo, então
ela disse somente ``Be''.

A Criança sorriu e disse:

``Certo, você deve me chamar somente de Be''.

Novamente, todos os animais disseram juntos como um grito:

``Certo, você deve me chamar somente de Be'', e então todos eles
repetiram as mesmas maneiras de mostrar sua alegria como antes.

May disse à Criança -- e de alguma forma sua voz pareceu muito, muito
alta, apesar de ela não ter intenção de sair assim, mas somente de
sussurrar:

``Oh, querido Be, eu desejaria muito que eles sempre continuassem
felizes e em paz dessa maneira. Não há meios de fazer isso?''

A Bela Criança abriu sua boca para falar, e todas as coisas vivas
colocaram suas garras, ou suas asas, ou suas barbatanas nos ouvidos para
ouvir com atenção.

Ela falou, e suas palavras pareciam cheias de som, mas muito suaves,
como o eco de um trovão distante vindo de águas longínquas nas asas da
música.

``Sabei, queridas crianças, e sabei vós todos que escutais -- haverá paz
na terra entre todas as coisas vivas quando os filhos dos homens
ficarem, por uma hora, em perfeito amor e perfeita harmonia um com o
outro. Lutai, oh!, lutai cada um de vós, para que assim o seja''.

Enquanto ele falou, ouviu"-se um silêncio solene e eles ficaram muito
quietos.

Então a Criança Maravilhosa pareceu flutuar dos braços de May e mover"-se
em direção ao mar. Todas as coisas vivas instantaneamente se apressaram
para formar uma fila dupla entre a qual ela passou.

May e Sibold seguiram"-no de mãos dadas. Ela esperou por eles na borda do
mar e então beijou"-os ambos.

Enquanto ele estava os beijando, o barco aproximou"-se da margem; a
âncora subiu a bordo; as velas brancas abriram"-se para cima; e uma brisa
fresca começou a soprar em direção de casa.

A Criança Maravilhosa moveu"-se para a proa e ali repousou. Sibold e May
subiram a bordo e tomaram seus lugares de antes; e depois de enviar
beijos com as mãos para todas as coisas vivas -- que estavam, nesse
momento, dançando juntas na clareira -- eles mantiveram seus olhos fixos
no Belo Garoto.

Quando se sentaram de mãos dadas, o barco moveu"-se suavemente, porém
muito rápido. A~encosta, com seus muitos lugares belos, parecia
deslizar, tornando"-se uma névoa turva à medida que rapidamente passavam.

Logo depois, eles viram seu próprio riacho e o grande Salgueiro
erguendo"-se acima de todas as outras árvores na margem.

O barco chegou a terra. A~Criança Maravilhosa, flutuando no ar, moveu"-se
em direção ao Desvão do Salgueiro.

Sibold e May seguiram"-na.

Ele entrou no Desvão; eles seguiram logo depois.

Quando a cortina folhada caiu atrás deles, o vulto da Criança
Maravilhosa tornou"-se cada vez mais turvo, até que, por fim, olhando"-os
amavelmente, e abanando suas pequeninas mãos, como se os abençoando, ela
pareceu desvanecer no ar.
\ \


Sibold e May ficaram sentados por um longo tempo, de mãos dadas,
pensando. Então, ambos se sentindo sonolentos, colocaram seus braços em
volta um do outro, e deitaram"-se para descansar.

Nessa posição dormiram novamente, com as Papoulas em volta deles.
