\chapterspecial{A criança maravilhosa}{}{}
 

Muito longe, à beira de um grande riacho que se estendia terra adentro
desde o mar infinito, havia uma tranquila vila.

Aqui os lavradores levavam uma vida feliz e próspera. Levantavam"-se
cedo, de forma que na manhã fria e gris ouviam a cotovia, totalmente
invisível nas alturas da manhã, cantando o hino matinal de que nunca se
esquece.

Quando o pôr do sol vinha furtivo, eles retornavam às suas casas,
felizes pelo que o resto do cair da noite lhes trazia.

No outono, quando a colheita devia ser feita, trabalhavam até tarde,
como eram capazes de fazer; pois, naquela época, o bom Sol e sua esposa,
a Lua, tinham o pacto de ajudar aqueles que trabalhassem na colheita.
Então o sol ficava no céu um pouco mais, a lua saía de sua cama no
horizonte um pouco mais cedo, e assim havia sempre luz para trabalhar.

A lua rubra, larga e cheia, que olha lá de cima os lavradores
trabalhando, é chamada de Lua da Colheita.

O Senhor da Mansão dessa vila pacífica era um homem muito bom e afável,
que sempre ajudava os pobres. Na hora da refeição, a porta de sua mansão
ficava aberta, e todos os que estavam com fome podiam entrar, se assim
escolhessem, tomar assento à mesa como hóspedes bem"-vindos.

Esse Senhor da Mansão tinha três filhos, Sibold e May, e um Garotinho
que havia acabado de chegar à casa e até então não tinha nome.

Sibold havia acabado de completar seu oitavo aniversário, e May estava a
dois meses de seu sexto. Eles gostavam muito um do outro -- como irmãos
e irmãs devem gostar -- e faziam todas as brincadeiras juntos. May
pensava que Sibold era muito grande e forte, e com qualquer coisa que
ele desejasse fazer ela sempre concordava.

Sibold adorava descobrir coisas e fazer explorações; e, em épocas
diferentes, as duas crianças já haviam estado em todos os domínios de
seu pai.

Eles tinham certos abrigos secretos dos quais ninguém sabia exceto eles
mesmos. Alguns eram lugares extraordinários e deleitáveis.

Um ficava no centro de um Carvalho oco, no qual viviam tantos esquilos
que, devido às suas idas e vindas, os galhos eram quase como ruas de uma
cidade.

Outro lugar ficava no topo de uma rocha, que só podia ser alcançado por
um caminho estreito entre altos arbustos de heras. Aqui havia um tipo de
cadeira grande cinzelada na rocha, capaz de acomodar os dois bem
ajustadamente; e para cá eles frequentemente levavam seu almoço e se
sentavam metade do dia, contemplando os topos das árvores onde, bem
longe na distância, a alva borda do horizonte se deitava no mar
cintilante.

Então, eles contavam um ao outro as coisas que pensavam, e o que
gostariam de fazer, e o que tentariam fazer quando crescessem.

Havia também outro lugar, que era o favorito entre todos.

Era embaixo de um grande Salgueiro Chorão. Essa era uma árvore vigorosa,
com muitas centenas de anos, que se erguia alta acima das outras árvores
que pontuavam a relva. Os longos galhos caíam tão espessos que, até
mesmo no inverno, quando as folhas tinham caído e os galhos estavam nus,
mal se podia ver dentro do oco que havia ali dentro.

Quando a nova roupagem da primavera voltava, a árvore toda, do cimo mais
alto até o solo musgoso do qual se erguia, tornava"-se uma massa compacta
e verde; e era difícil entrar nela, mesmo se sabendo o caminho.

Em certo lugar, um dos galhos longos tinha sido quebrado por uma grande
tempestade, a qual, muito tempo atrás, havia posto abaixo várias árvores
da floresta; mas os galhos que pendiam perto dele lançaram novos brotos
para preencher o espaço vazio, e, assim, a abertura foi coberta com
ramos finos ao invés de galhos fortes.

No verão, as folhas cobriam tudo com uma massa verde; mas aqueles que
conheciam a abertura podiam empurrar de lado os ramos e, assim, entrar
no caramanchão.

Era um caramanchão lindíssimo. Não importava o quão forte o sol
brilhasse lá fora, dentro era fresco e agradável. Do chão ao topo, até o
próprio dossel em que os galhos negros, encontrando"-se, formavam uma
massa escura, tudo era de um verde delicado, pois a luz lá de fora
entrava pelo meio das folhas suave e docemente.

Sibold e May pensavam que o mar devia parecer assim às Sereias, que
cantam e penteiam seus longos cabelos com pentes dourados, nas
profundezas frias do oceano.

Na relva ao redor dessa grande árvore havia muitos canteiros de belas
flores. Os Ásteres de faces largas e coloridas, fitando diretamente o
sol lá no alto sem mesmo piscar, acima e ao redor dos quais esvoaçavam
adoráveis borboletas, com suas asas semelhantes aos arco"-íris ou aos
pavões ou aos pores do sol ou ao que há de mais deslumbrante. A~doce
Reseda, sobre a qual pairavam abelhas com um zumbido agradecido. Os
Amores"-Perfeitos, com suas faces grandes e delicadas tremendo em seus
caules delgados. As Tulipas, abrindo suas bocas ao sol e à chuva; pois a
Tulipa é uma flor gulosa, que abre demais sua boca, até que, de tão
aberta, sua cabeça se desfaz em pedaços e ela morre. Os Jacintos, com
seus muitos sinos agrupados num galho -- como uma grande festa de
família. Os imensos Girassóis, com suas faces pendentes a brilhar como
filhos do próprio pai, o Sol.

Havia também grandes Papoulas, com folhas espraiadas e descuidadas,
caules grossos e suculentos, e grandes flores escarlates, que erguiam e
pendiam como bem quisessem, e que pareciam muito livres e descuidadas e
independentes.

Tanto Sibold quanto May amavam essas Papoulas, e todos os dias iam
olhá"-las. Nos canteiros da relva musgosa, onde se erguia o grande
Salgueiro, elas cresciam até atingir um tamanho enorme, tornavam"-se tão
altas que, quando Sibold e May ficavam de mãos dadas ao lado do
canteiro, as grandes Papoulas se elevavam acima deles, e Sibold, subindo
na ponta dos pés, sequer era capaz de alcançar as flores escarlates.

Um dia, depois do desjejum, Sibold e May levaram consigo seu almoço, e
saíram para passar o dia juntos, passeando entre os bosques, pois isso
era uma festa para eles. Um pequenino irmão Garotinho havia chegado à
casa, e todos estavam ocupados arrumando coisas para ele. As crianças
haviam"-no visto somente por um instante.

De mãos dadas, Sibold e May percorreram todos os seus lugares favoritos.
Eles olharam a toca no Carvalho, e diziam ``Como vai o senhor?'' a todo
esquilo que vivia na árvore, e contavam"-lhes sobre o novo Bebê que havia
chegado ao lar. Depois, foram à rocha, e sentaram"-se juntos no assento,
e observaram o mar distante.

Permaneceram ali por um tempo sob a luz do sol quente, e falaram do
pequeno e querido irmãozinho bebê que haviam visto. Ficaram a se
perguntar de onde ele tinha vindo, e fizeram um plano: procurariam e
procurariam até que eles também encontrassem um bebê. Sibold disse que
ele devia ter vindo lá do mar, colocado no canteiro de salsa pelos
Anjos, de maneira que uma babá pudesse encontrá"-lo ali e levá"-lo à pobre
mãe deles, que estava enferma e precisava de consolação. Assim, eles
matutaram como seriam capazes de partir para além do mar, e planejaram
que algum dia o barco de Sibold seria aumentado, e eles entrariam nele e
partiriam pelos mares, e procurariam outro bebezinho só para eles.

Depois de um tempo, eles se cansaram de se sentar no sol quente; então,
deixaram o lugar e, de mãos dadas, perambularam até chegar à relva plana
onde o grande Salgueiro se erguia, e onde os canteiros de flores faziam
o ar parecer cheio de cor e de perfume.

De mãos dadas eles caminharam, olhando as borboletas, e as abelhas, e os
pássaros, e as belas flores.

Num canteiro viram que uma nova flor que aparecera. Sibold a conhecia e
contou a May que era um Lírio Asiático; ela teve medo de se aproximar,
mas então ele lhe disse que a flor não a machucaria, pois era somente
uma flor.

À medida que caminhavam, Sibold colhia algumas flores de cada canteiro e
as dava à irmã; quando eles estavam prestes a se afastar do Lírio
Asiático, ele arrancou a flor, e, porque May tinha medo de carregá"-la,
ele mesmo a levou.

Por fim, chegaram ao grande canteiro de Papoulas. As flores pareciam tão
brilhantes e frescas, devido a sua cor vibrante, e tão despreocupadas,
que May e Sibold pensaram, ambos ao mesmo tempo, que gostariam de pegar
muitas delas e levá"-las consigo para o interior do Caramanchão do
Salgueiro, pois estavam indo comer lá e queriam que o lugar estivesse
tão alegre e bonito quanto possível.

Mas, antes, voltaram ao Carvalho para recolher um monte de folhas, pois
Sibold sugeriu que fizessem do novo irmão bebê o Rei do Banquete, e que
trançassem para ele uma coroa de carvalho. Como o próprio bebê não
estaria lá, eles colocariam a coroa onde a pudessem ver bem.

Quando chegaram ao Carvalho, May exclamou:

``Oh, olhe, Sibold, olhe, olhe!''

Sibold olhou, e viu que em quase todos os galhos havia um monte de
esquilos sentados dois a dois, com suas caudas peludas sobre suas
costas, comendo nozes tão ávidos quanto podiam.

Quando os esquilos os viram, não tiveram medo, pois as crianças nunca
lhes haviam feito mal algum. Deram todos juntos uma grasnada esquisita e
um pulinho engraçado. Sibold e May começaram a rir, mas não gostavam
incomodar os esquilos, então juntaram tantas folhas de carvalho quanto
quiseram e voltaram ao canteiro de Papoulas.

``Agora, Sibold, querido'', disse May, ``precisamos pegar muitas
Papoulas, pois o querido Bê gosta muito delas''.

``Como você sabe?'', perguntou Sibold.

``Porque ele deve gostar'', ela respondeu. ``Você e eu gostamos, e ele é
nosso irmão, então é claro que gosta''.

Então Sibold colheu muitas Papoulas, e algumas delas ele apanhou junto
com muitas folhas verdes e frescas ainda presas aos caules, até que
ambos ficaram com os braços cheios. Então, juntaram todas as outras
flores e entraram no Caramanchão do Salgueiro para comer. Sibold foi à
fonte que nascia no jardim e, atravessando"-o, corria até o mar. Ali ele
encheu seu gorro com água, trouxe"-o tão devagar quanto pôde para não
derramar muito líquido, e voltou ao caramanchão. Levantando os galhos,
May fez uma abertura na folhagem quando ele chegou; depois que ele
passou, ela os deixou cair novamente. Quando a cortina de folhas estava
pendurada toda em volta deles, as duas crianças ficaram sozinhas no
Caramanchão do Salgueiro.

Então se puseram a trabalhar para adornar sua cabana folhada com as
flores. Eles as enrolaram em torno dos galhos pendentes, e fizeram uma
guirlanda, que colocaram em volta do tronco da árvore. Por toda parte,
colocaram as Papoulas no lugar mais alto que podiam alcançar, e então
Sibold segurou May no alto enquanto ela enfiava os Lírios Asiáticos numa
fissura no tronco de árvore acima de todas as outras flores.

Então as crianças se sentaram para comer. Os dois estavam muito cansados
e com muita fome, e apreciaram muito o descanso e a comida. Havia
somente uma coisa que eles queriam, e essa coisa era o novo Irmãozinho
Bebê, para que pudessem fazer dele o rei do banquete.

Quando a refeição terminou, eles se sentiram muito cansados; então,
deitaram"-se juntos, com suas cabeças uma no ombro do outro e seus braços
entrelaçados; e ali eles foram dormir com as Papoulas escarlates
acenando em todo o entorno.

-

Depois de certo tempo, eles não estavam mais dormindo. Não parecia ser
mais tarde, mas sim de manhã cedo. Nenhum deles se sentia nem um pouco
sonolento ou cansado; ao contrário, ambos queriam partir para uma
expedição mais longa do que nunca.

``Venha ao riacho'', disse Sibold, ``e vamos sair daqui com meu barco''.

May levantou"-se, e eles abriram a porta de folhas e saíram. Desceram ao
riacho, e lá encontraram o barco de Sibold com todas as suas velas
armadas.

``Vamos entrar'', disse Sibold.

``Por quê?'', perguntou May.

``Porque assim podemos velejar um pouco'', ele respondeu.

``Mas ele não vai nos aguentar; é pequeno demais'', disse May, que
estava um tanto com medo de velejar, mas não queria dizê"-lo.

``Vamos tentar'', disse seu irmão. Ele tomou a corda que amarrava o
barco à margem e puxou"-a para si. A~linha parecia muito longa, e Sibold
parecia estar puxando"-a já por um bom tempo. De qualquer maneira, o
barco por fim chegou. À~medida que se aproximava, ficava cada vez maior,
até que, quando tocou a margem, eles viram que era grande o bastante
para aguentar os dois.

``Anda, vamos entrar'', disse Sibold.

De alguma forma, May não sentia mais medo. Entrou no barco e descobriu
que ali havia almofadas de seda da cor das flores de Papoula. Então
Sibold entrou, e soltou a corda que amarrava o barco à margem. Sentou"-se
na popa e segurou o leme em sua mão; May se sentou numa almofada no
fundo do barco e se segurou nas bordas.

As velas brancas inflaram com uma brisa suave, e eles começaram a se
afastar da margem; as ondinhas se agitavam contra a proa do barco. May
ouvia o bate"-bate repetitivo delas quando tocavam a proa, e então se
deitou.

O sol luzia muito brilhante. A água estava tão azul quanto o céu e tão
límpida que as crianças podiam olhar para baixo e ver seu fundo, onde os
peixes estavam a nadar rapidamente de lá para cá. Ali, também, as
plantas e as árvores que crescem sob a água estavam abrindo e fechando
seus galhos; e as folhas estavam se movendo como aquelas das árvores
terrestres quando sopra o vento.

Por certo tempo, o barco se afastou da terra, até que eles perderam de
vista o alto Salgueiro que sobressaía acima das outras árvores. Então o
barco pareceu se aproximar novamente da margem, e avançou assim, sempre
tão perto dela que as crianças podiam ver muito claramente tudo o que
havia lá.

A margem era muito variada; e cada momento mostrava algo novo e
belo…

Agora era uma rocha saliente, toda coberta de plantas rastejantes cujas
flores quase tocavam a água.

Agora era uma praia na qual a areia branca reluzia e resplandecia à luz
do sol, e na qual as ondas faziam um sussurro agradável quando corriam
pela margem acima e depois voltavam pela margem abaixo -- como se
brincassem de ``pega"-pega''.

Agora árvores escuras, de densa folhagem, pendiam sobre a água; mas,
atravessando seu breu, fragmentos de luz cintilavam ao longe, quando o
sol, passando por algumas aberturas, raiava dentro da clareira sob as
copas.

Por sua vez, havia lugares onde a grama, tão verde quanto esmeralda,
seguia em declive até a borda da água, e onde as flores de Prímulas e
Ranúnculos que cresciam à margem, debruçando"-se sobre ela, quase
beijavam as ondinhas que iam encontrá"-los.

Então havia lugares onde grandes Lilases, com o sopro de seus cachos de
flores rosas e brancas, tornavam o ar doce em toda a vasta cercania, e
onde os Laburnos pareciam despejar torrentes de ouro vindas da
abundância das flores que pendiam de seus retorcidos galhos verdes.

Havia também grandes Palmeiras, com suas folhas largas, fazendo uma
sombra fresca na terra abaixo. Grandes Coqueiros, em cujos troncos
tropas de macacos ficavam subindo para reunir cocos que eles arrancavam
e jogavam para baixo. Aloés com grandes caules carregados de flores
púrpuras e douradas -- que estavam completando cem anos, única ocasião
em que os aloés florescem.

Havia Papoulas tão grandes quanto árvores, e Lírios cujas flores eram
maiores que cabanas.

As crianças gostaram de todos esses lugares, mas, de repente, chegaram a
um lugar em que havia um terreno de grama esmeralda ensombrado por
árvores gigantescas. Em volta, crescia ou pendia ou se agrupava todo
tipo de flora vicejante. Canas"-de"-Açúcar altas brotavam da beirada de um
pequeno córrego que fluía sobre um leito de pedras brilhantes como
joias. Palmeiras elevavam suas cabeças eminentes, e plantas com grandes
folhas se erguiam e produziam sombras até mesmo na penumbra. Perto havia
uma fonte cristalina que borbotava formando um pequeno córrego de onde
as Canas"-de"-Açúcar se erguiam.

Quando viram esse lugar, ambas as crianças exclamaram: ``Oh! Que bonito!
Vamos parar aqui''.

O barco pareceu entender o desejo deles, pois, sem que o leme fosse
tocado, virou"-se e flutuou suavemente para a margem.

Sibold desceu do barco e levantou May para levá"-la até a terra. Ele
pretendia atracar; mas, no instante em que May saiu do barco, todas as
velas se dobraram por si mesmas, a âncora pulou para fora e, antes que
fosse possível fazer qualquer coisa, o barco estava ancorado perto da
margem.

Sibold e May deram"-se as mãos e perambularam juntos pelo lugar, olhando
para tudo.

De repente, May disse, num sussurro:

``Oh, Sibold, este lugar é tão gostoso, será que há Salsa aqui?''

``Por que você quer Salsa?'', ele perguntou.

``Porque, se houver um bom canteiro de Salsa, talvez possamos encontrar
um Bebê… E, oh!, Sibold, quero \emph{muito} um Bebê''.

``Muito bem, então, vamos procurar'', disse seu irmão. ``Parece haver
todo tipo de planta aqui; e se há \emph{todo} tipo de planta, com
certeza \emph{tem que} haver Salsa''. Pois Sibold era muito lógico.

Então, as duas crianças caminharam por toda a várzea gramada,
procurando; e, logo depois, como previsto, encontraram sob as folhas
espalhadas de uma Cidra um grande canteiro de Salsa -- as maiores Salsas
que já haviam visto.

Sibold ficou bem satisfeito, e disse: ``Isso se parece com Salsa. Sabe,
May, sempre me intrigou como um Bebê, que é muito maior do que a Salsa,
possa estar escondido nela; e ele deve ficar bem escondido, pois muitas
vezes saio para olhar o canteiro de casa e nunca consigo achar um bebê,
mas a enfermeira sempre acha quando vai procurar. Só que ela não procura
quase nunca. Sei que, se eu fosse tão sortudo quanto ela, ficaria sempre
procurando''.

May viu que o desejo de encontrar um bebê se tornou tão forte nela que
disse novamente:

``Oh, Sibold, eu desejo \emph{tanto} um Bebê; \emph{espero} que
encontremos um''.

Assim que ela falou, ouviu"-se um som estranho -- um tipo de risada
muito, muito leve -- como um sorriso transformado em música.

May ficou surpresa e, por um momento, não pensou em fazer coisa alguma;
ela simplesmente apontou, e disse:

``Olhe, olhe!''

Sibold correu na direção da risada e levantou a folha de um enorme pé de
Salsa; e ali -- oh, alegria de alegrias! -- estava deitado o Bebê mais
adorável que já fora visto.

May ajoelhou"-se ao lado dele, levantou"-o, começou a balançá"-lo e cantou
``Nana nenê'', enquanto Sibold olhava complacente. Entretanto, depois de
uns instantes, ele ficou impaciente e disse:

``Veja bem, entende, eu encontrei esse Bebê; ele pertence a mim''.

``Oh, por favor'', disse May, ``eu o ouvi primeiro. Ele é meu''.

``Ele é meu'', disse Sibold; ``Ele é meu'', disse May; e ambos começaram
a ficar um pouco nervosos.

De repente, ouviram um gemido baixo -- um tipo de som que parecia música
com dor de dente. Ambas as crianças olharam para baixo alarmadas, e
viram que o pobre Bebê estava morto.

Os dois ficaram horrorizados e começaram a chorar; e pediram perdão um
ao outro, e prometeram que nunca, nunca mais iriam ficar nervosos.
Depois que fizeram isso, a Criança abriu seus olhos, olhou para eles
gravemente, e disse:

``Agora, nunca briguem ou fiquem nervosos. Se ficarem nervosos de novo,
qualquer um dos dois, eu morrerei, sim, e serei enterrado também, antes
que vocês possam dizer `raquetes'\,''.

``É mesmo, Bê'', disse May, ``nunca, nunca ficarei brava de novo. Ao
menos, eu tentarei não ficar''.

Disse Sibold:

``Garanto"-lhe, senhor, que, esteja eu sob qualquer provocação,
resultante de qualquer concatenação das circunstâncias, jamais serei
culpado da \emph{malfaisance} da raiva''.

``Como ele fala bonito'', disse May; e o Bebê acenou para ele com a
cabeça, revelando familiaridade, como se dissesse:

``Tudo bem, velho, nós nos entendemos''.

Então, por um tempo, todos eles ficaram bem quietos. De repente, o Bebê
virou seus olhos azuis para May e disse:

``Por favor, mãezinha, cantaria para mim?''

``O que você gostaria, Bê?'', perguntou May.

``Oh, qualquer coisinha, algo de dramático'', ele respondeu.

``Algum estilo em particular?'', perguntou May.

``Não, obrigado; qualquer coisa que venha a calhar. Prefiro algo simples
-- alguma coisinha elementar, como, por exemplo, qualquer cançãozinha
começando com uma escala cromática em quintas e oitavas consecutivas,
\emph{pianissimo} -- \emph{rallentando} -- \emph{excellerando}
--\emph{crescendo} -- até chegar a uma dissonância na dominante da nona
bemol diminuta''.

``Oh, por favor, Bê'', disse May, muito humildemente, ``ainda não sei
nada sobre isso. Ainda estou nas escalas e, perdão, não sei do que se
trata tudo isso''.

``Olhe, e você verá'', disse a Criança, e tomou um pedaço de graveto e
escreveu uma música na areia.

``Ainda não sei'', disse May.

Bem naquele momento, um pequeno animal marrom"-amarelado apareceu na
clareira caçando um rato. Quando ficou defronte a eles de repente
disparou para longe como o som de uma pistola.

``Agora você sabe?'', perguntou a Criança.

``Não, querido Bê, mas não importa'', ela respondeu.

``Muito bem, querida'', disse a Criança, beijando"-a, ``qualquer coisa
que lhe agradar, só deixe vir diretamente de seu coraçãozinho amável'',
e beijou"-a novamente.

Então May cantou algo muito doce e belo -- tão doce e belo que a fez
chorar, e também Sibold, e o Bebê. Ela não conhecia a letra, e não
conhecia a melodia, e tinha somente uma noção bem vaga do que falava;
mas era tudo muito, muito bonito. Durante todo o tempo em que cantava,
ela ficou acalentando Bebê, e ele colocou seus bracinhos gordos em volta
do pescoço dela, e a amou muito.

Quando ela terminou de cantar, a Criança disse:

``Chlap, Chlap, Chlap, M"-chlap!''

``O que ele quer dizer?'', ela perguntou a Sibold, desconfortada, pois
viu que o Bebê queria algo.

Naquele momento, uma bela Vaca colocou sua cabeça por sobre os arbustos
e disse: ``Muu"-uu"-uu''. A~Bela Criança bateu palmas; assim também May,
que disse:

``Oh, eu já sei. Ele quer ser alimentado''.

A Vaca entrou sem ser convidada, e Sibold disse:

``Acho que sim, May, melhor eu tirar leite dela''.

``Por favor, sim, querido'', disse May. E~começou a ninar novamente o
Bebê, a beijá"-lo, a acalentá"-lo, e a lhe contar que logo ia ser
alimentado.

Enquanto se ocupava com isso, ela ficou sentada de costas para Sibold.
Mas o Bebê estava olhando para a ordenhação, com seus olhos azuis
dançando de alegria. Subitamente, começou a rir, rir tanto que May olhou
em volta para ver do que ele estava rindo. Ali estava Sibold tentando
ordenhar a Vaca puxando seu rabo.

A Vaca não parecia se importar com ele, e continuou a pastar.

``Eia, Dona'', disse Sibold. A~Vaca começou a saltitar.

``Oh, ora essa'', disse Sibold, ``vamos! Apresse"-se e nos dê um pouco de
leite; o Bê quer um pouco''.

A Vaca lhe respondeu:

``O querido Bê não deve desejar nada''.

May pensou que era muito estranho a Vaca poder falar; mas, como Sibold
não pareceu achar isso estranho, segurou a língua.

Sibold começou a discutir com a Vaca: ``Mas, convenhamos, Senhora Vaca,
se ele não deve desejar nada, por que a senhora o faz desejar?''

A Vaca respondeu: ``Não me culpe. A~culpa é sua. Tente de outro jeito'',
e começou a rir tanto quanto podia.

Sua risada era muito engraçada, a princípio muito alta, mas gradualmente
ficando mais e mais parecida com a risada da Criança, até que May não as
conseguia distinguir. Então, a Vaca parou de rir, mas a Criança
continuou.

``Do que está rindo, Bê?'', May perguntou, pois não tinha ideia se sabia
mais de ordenhação que Sibold. Ela achou isso muito engraçado, pois
sabia que muitas vezes havia visto as vacas serem ordenhadas em casa.

O Bebê falou: ``Não é assim que se ordenha uma vaca''.

Então Sibold começou a levantar e abaixar o rabo da Vaca como a haste de
uma bomba; mas o Bebê riu ainda mais.

Subitamente, sem saber como isso veio a acontecer, May viu derramando
leite de um regador em cima do Bebê todo, que estava deitado no chão,
com Sibold segurando sua cabeça. O~Bebê estava gritando satisfeito e
rindo como um louco; e quando o regador ficou vazio, ele disse:

``Muito obrigado aos dois. Nunca apreciei tanto um jantar em minha
vida''.

``Esse é um Bê muito querido e estranho!'', disse May, em sussurros.

``Muito'', disse Sibold.

Enquanto falavam, veio um som terrível de entre as árvores, muito, muito
longe a princípio, mas que se aproximava mais e mais a cada momento. Era
como gatos que estavam tentando imitar um trovão. O~barulho veio
estrondeando através das árvores.

``Meiau"-u-room"-r-p"-sss. Rarkrrau"-iau"-p-ss''.

May ficou muito assustada; e Sibold também, mas ele não iria admitir.
Sentiu que tinha de proteger sua Irmãzinha e o Bebê, então se pôs entre
os dois e o lugar de onde vinha o som. May abraçou forte a Criança, e
lhe disse: ``Não tenha medo, querido Bê. Nós não vamos deixar ele tocar
em você''.

``O que é `ele'?'', perguntou o Bebê.

``Eu não sei, Bê'', ela respondeu. ``Gostaria de saber. Lá vem ele
agora''; pois, exatamente naquele instante, um Tigre grande e bravo
apareceu nos topos das árvores mais altas e ficou lá, fitando"-os
furiosamente com seus grandes olhos verdes e flamejantes.

May olhou para essa coisa terrível com seus olhos arregalados de terror;
mas, mesmo assim, abraçou o Bebê cada vez mais forte. Ficou olhando para
o Tigre, e viu que ele não estava mirando nem ela nem Sibold, mas sim o
Bebê. Isso a deixou mais assustada do que nunca, e ela o agarrou ainda
mais forte. Enquanto olhava, no entanto, percebeu que os olhos do Tigre
ficavam menos bravos a cada instante que passava, até que, por fim,
ficaram tão gentis e mansos quanto os de seu gatinho malhado favorito.

Então o Tigre começou a ronronar. O~ronronar era como o rom"-rom de um
gato, mas tão alto que soava como tambores. Entretanto, ela não se
importou com isso, pois, apesar de o som ser alto, era como se buscasse
ser gentil e carinhoso. Então o Tigre chegou perto e agachou diante da
Criança Maravilhosa, e lambeu suas mãozinhas gordas com sua grande e
áspera língua vermelha, mas muito suavemente. O~Bebê riu, e acariciou o
grande focinho do Tigre, e puxou os longos bigodes eriçados, e disse:

``Gii, gii''.

O Tigre passou a se comportar de maneira muito engraçada. Ele se deitou
de costas e rolou ao redor, depois ficou em pé e ronronou mais alto que
nunca. Sua grande cauda se ergueu diretamente para cima, com a ponta se
movendo ao redor e derrubando aqui e ali um monte de uvas que pendiam da
árvore acima. Parecia inundado de alegria, veio e agachou novamente
diante da Criança, e ronronou em volta dele em grande estado de alegria.
Finalmente, deitou"-se, sorrindo e ronronando, e zelando pela Criança,
como se estivesse de guarda.

Logo depois veio de longe outro som terrível. Era como um grande Gigante
sibilando; e era mais alto do que um trem, e com mais vozes que um bando
de gansos. Havia também o som de galhos se partindo, de esmagamento da
vegetação rasteira; e havia um som terrível de algo sendo arrastado,
diferente de tudo que eles já tinham ouvido.

Novamente Sibold se prostrou entre o som e May, que mais uma vez segurou
o Bebê para protegê"-lo do mal.

O Tigre se levantou e arqueou suas costas como um gato bravo, e ficou
pronto para atacar qualquer coisa que viesse.

Então ali apareceu, sobre os topos das árvores, a cabeça de uma enorme
Serpente, com olhos miúdos que brilhavam como fagulhas de fogo e duas
grandes mandíbulas abertas. Essas mandíbulas eram tão grandes que
realmente parecia que toda a cabeça do animal tinha se aberto em duas; e
entre as mandíbulas havia uma grande língua bifurcada que parecia cuspir
veneno. Atrás dessa cabeça monstruosa apareceram as enormes curvas do
corpo da Serpente, movendo"-se continuamente. O~Tigre rugiu como que a
ponto de atacar; mas, de repente, a Serpente baixou sua cabeça em
submissão. Estava fitando a Criança Maravilhosa; e, quando May olhou,
também viu que o pequenino Bebê estava apontando para baixo, como que
dando ordens à Serpente a seus pés. Então o Tigre, com um rosnado baixo
e depois um ronronado contente, voltou ao seu lugar para vigiar e ficar
de guarda. A~grande Serpente veio suavemente e se enrolou dentro da
clareira, e também parecia como se estivesse vigiando e guardando a
Criança Maravilhosa.

Novamente veio outro som terrível. Dessa vez, vinha do ar. Grandes asas
pareciam bater com um som mais alto do que um trovão; e, de longe, o ar
foi escurecido por uma portentosa Ave de Rapina que lançava uma sombra
sobre a terra com suas asas abertas.

Quando a Ave de Rapina desceu num mergulho, o Tigre se levantou
novamente e arqueou suas costas como que prestes a pular para
encontrá"-la no ar, e a Serpente levantou seu corpo cheio de curvas e
abriu suas mandíbulas como se prestes a dar o bote.

Mas, quando a Ave viu a Criança, ela também se tornou menos feroz, e
ficou suspensa no ar a meia altura, com sua cabeça inclinada, como que
em submissão. Logo, a Serpente se enrolou como antes, o Tigre voltou a
vigiar e ficar de guarda, e a Ave de Rapina pousou na clareira, também
vigiando e fazendo guarda.

May e Sibold começaram a observar maravilhados o Belo Garoto, ante a
quem esses monstros faziam reverência; mas eles não conseguiam ver nada
de estranho.

Novamente, houve outro som terrível -- dessa vez lá do mar --, uma
arremetida e um assovio, como se alguma coisa gigante estivesse
chicoteando a água.

Olhando em torno, as crianças viram dois monstros se aproximando. Um
Tubarão e um Crocodilo. Eles surgiram do mar e vieram para a terra. O~Tubarão estava pulando, batendo com sua cauda e rangendo sua tripla
fileira de grandes dentes. O~Crocodilo estava rastejando com seus
grandes pés e pernas curtas e curvas, e sua boca terrível estava abrindo
e fechando, batendo seus grandes dentes.

Quando os dois se aproximaram, o Tigre e a Serpente e a Ave de Rapina se
ergueram todos para proteger a Criança; mas, quando os recém"-chegados
viram o Bebê, também prestaram reverência e mantiveram vigia e guarda --
o Crocodilo rastejando na praia, e o Tubarão se movendo para lá e para
cá na água -- iguais a sentinelas.

Novamente, May e Sibold olharam para a Bela Criança e se espantaram.

Mais uma vez, houve um som terrível, mais horrível do que os anteriores.

A terra pareceu tremer e um som profundo e abafado veio de muito abaixo.
Então, um pouco longe, uma montanha se ergueu de repente; seu cume se
abriu, e dali estouraram, com um som mais alto do que o de uma
tempestade, fogo e fumaça. Grandes volumes de vapor preto se ergueram e
ficaram a pairar, uma nuvem preta pendia acima das cabeças. Pedras em
brasa, de um tamanho enorme, foram atiradas para o alto e caíram
novamente na cratera, e sumiram. Pelas encostas da montanha rolavam
torrentes de lava incandescente, e fontes de água fervente irrompiam por
todo lado.

Sibold e May ficaram mais assustados do que nunca, e May segurou o
querido Bebê bem junto ao seu peito.

O troar da montanha flamejante ficou cada vez mais alto, a lava ardente
jorrava densa e rápida, e da cratera se ergueu a cabeça de um feroz
Dragão, com olhos como carvão incandescente e dentes como línguas de
fogo.

Então o Tigre e a Serpente e a Ave de Rapina, e o Crocodilo e o Tubarão,
todos se preparam para defender a Criança Maravilhosa.

Mas quando o feroz Dragão viu o Garoto, também ele foi dominado, e
rastejou humildemente para fora da cratera em chamas.

Então, a montanha furiosa afundou novamente para dentro da terra, a lava
incandescente desapareceu; e o Dragão permaneceu com os outros para
vigiar e ficar de guarda.

Sibold e May ficaram mais impressionados do que nunca, e olharam para o
Bebê com uma curiosidade ainda maior. De repente, May disse a seu irmão:

``Sibold, quero cochichar algo a você''.

Sibold inclinou sua cabeça, e ela sussurrou muito baixo em seu ouvido:

``Eu acho que Bê é um Anjo!''

Sibold olhou pasmo para o Bebê e respondeu:

``Eu também acho, querida. O~que vamos fazer?''

``Não sei'', disse May, ``espero que ele não fique bravo conosco por o
chamarmos de `Bê'\,''.

``Espero que não'', disse Sibold.

May pensou por um momento, e seu rosto se iluminou com um sorriso
contente, quando ela disse:

``Ele não ficará bravo, Sibold. Como você sabe, nós o acolhemos
inadvertidamente''.

``É bem verdade'', disse Sibold.

Enquanto estavam falando, todos os tipos de animais e pássaros e peixes
estavam vindo à clareira, andando de braços dados do jeito que
conseguiam -- pois nenhum deles tinha braços. Um Leão e um Carneiro
vieram primeiro, e os dois se curvaram diante da Criança, e depois se
foram e se deitaram juntos. Então veio uma Raposa e um Ganso; e depois
um Gavião e um Pombo; e depois um Lobo e outro Carneiro; e depois um
Cachorro e um Gato; e depois outro Gato e um Rato; e depois outra Raposa
e uma Cegonha; e uma Lebre e uma Tartaruga; e um Lúcio e uma Truta; e um
Pardal e uma Minhoca; e muitos, muitos outros, até que toda a clareira
estivesse cheia de coisas vivas, todas em paz uma com a outra.

Todos se sentaram ao redor da clareira em pares, e todos olhavam para a
Criança Maravilhosa.

May sussurrou novamente para Sibold:

``Acho que, se ele for um anjo, devemos ser muito respeitosos com ele''.

Sibold anuiu, mostrando que concordava com ela; então, ela aconchegou o
Bebê mais perto de si e disse:

``Por favor, senhor Bê, sentados assim, eles não parecem todos bons e
belos?''

A Bela Criança sorriu docemente quando respondeu:

``Belos e doces eles parecem''.

May disse novamente:

``Eu gostaria que eles sempre fossem assim, e nunca brigassem ou
discordassem de forma alguma, querido Bê. Oh! Peço perdão. Digo, Senhor
Bê''.

A Criança perguntou a ela:

``Por que pede meu perdão?''

``Porque lhe chamei de Bê, ao invés de Senhor Bê''.

O Garoto perguntou novamente:

``Por que você deveria me chamar de Senhor Bê?''

May não quis dizer ``Porque você é um Anjo'' da forma como gostaria de
ter feito, então, ela aconchegou a Criança mais perto de si e sussurrou
em seu pequeno ouvido róseo:

``Você sabe''.

A criança colocou seus bracinhos em volta do pescoço dela e beijou"-a, e
disse, bem baixo e bem docemente, palavras que por toda sua vida ela
jamais esqueceu:

``Eu sei. Seja sempre carinhosa e doce, querida criança, e até mesmo os
Anjos conhecerão seus pensamentos e escutarão suas palavras''.

May sentiu"-se muito feliz. Olhou para Sibold, que se inclinou e
beijou"-a, e chamou"-a de ``doce irmãzinha''; e todos os animais, em
pares, e todos aqueles terríveis que estavam de guarda, todos disseram
juntos, como uma aclamação:

``Certo!''

Então fizeram uma pausa e emitiram todos juntos, seguindo a roda, cada
um dos sons que cada um usava para mostrar que estava feliz. Primeiro
todos ronronaram, e depois todos grasnaram, e depois todos cacarejaram,
e grunhiram, e bateram as asas e sacudiram suas caudas.

``Oh, que bonito!'', disse May novamente, ``olhe, querido Bê!'' Ela
estava prestes a dizer Senhor quando a Criança levantou seu dedo, então
ela disse somente ``Bê''.

A Criança sorriu e disse:

``Certo, você deve me chamar somente de Bê''.

Novamente, todos os animais disseram juntos, como um grito:

``Certo, você deve me chamar somente de Bê'', e então todos eles
repetiram juntos e em roda cada uma das maneiras de expressar sua
alegria, tal como antes.

May disse à Criança -- e de alguma forma sua voz pareceu muito, muito
alta, apesar de ela não ter intenção de sair assim, mas somente de
sussurrar:

``Oh, querido Bê, eu queria tanto que eles sempre continuassem felizes e
em paz desse jeito. Não há meios de fazer isso?''

A Bela Criança abriu sua boca para falar, e todas as coisas vivas
colocaram suas garras, ou suas asas, ou suas barbatanas nos ouvidos para
ouvir com atenção.

Ele falou, e suas palavras pareciam cheias de som, mas muito suaves,
como o eco de um trovão distante, vindo de águas longínquas nas asas da
música.

``Sabei, queridas crianças, e sabei vós todos que escutais -- haverá paz
na terra entre todas as coisas vivas quando os filhos dos homens
ficarem, por uma hora, em perfeito amor e em perfeita harmonia um com o
outro. Lutai, oh!, lutai cada um de vós, para que assim o seja''.

Enquanto ele falou, ouviu"-se um silêncio solene, e os dois ficaram muito
quietos.

Então a Criança Maravilhosa pareceu sair flutuando dos braços de May e
se mover em direção ao mar. Todas as coisas vivas instantaneamente se
apressaram para formar uma fila dupla entre a qual ela passou.

May e Sibold seguiram"-na de mãos dadas. Ela esperou por eles na borda do
mar e então beijou a ambos.

Enquanto ela os beijava, o barco se aproximou da margem; a âncora subiu
a bordo; as velas brancas se abriram para cima; e uma brisa fresca
começou a soprar em direção de casa.

A Criança Maravilhosa foi para a proa e ali repousou. Sibold e May
subiram a bordo, e tomaram seus lugares de antes; e depois de enviar
beijos com as mãos para todas as coisas vivas -- que estavam, nesse
momento, dançando todas juntas na clareira --, mantiveram seus olhos
fixos no Belo Garoto.

Quando os dois se sentaram de mãos dadas, o barco se moveu suavemente,
mas muito rápido. À~medida que passavam velozmente, a encosta, com seus
muitos lugares bonitos, parecia deslizar e tornar"-se uma névoa turva.

Logo depois, eles viram seu próprio riacho, e o grande Salgueiro
erguendo"-se acima de todas as outras árvores à margem.

O barco chegou a terra. A~Criança Maravilhosa, flutuando no ar, moveu"-se
em direção ao Caramanchão do Salgueiro.

Sibold e May seguiram"-na.

Ela entrou no caramanchão; eles vieram logo depois.

Quando a cortina folhada caiu atrás deles, o vulto da Criança
Maravilhosa foi ficando cada vez mais indiscernível, até que, por fim,
olhando"-os amavelmente, e abanando suas pequeninas mãos, como que os
abençoando, ela pareceu desvanecer no ar.

-

Sibold e May ficaram sentados por um longo tempo, de mãos dadas,
pensando. Então, ambos se sentindo sonolentos, colocaram seus braços um
em volta do outro, e deitaram"-se para descansar.

-

Nessa posição, dormiram de novo, com as Papoulas ao seu redor.
