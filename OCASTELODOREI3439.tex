\chapterspecial{O castelo do rei}{}{}
 

Claribel vivia em paz e feliz com seu pai e sua mãe, desde o tempo em
que era um bebezinho até quando, aos dez anos, foi para a escola.

Seus pais eram pessoas boas e adoráveis, que amavam a verdade e tentavam
sempre andar no caminho dos justos. Eles ensinaram a Claribel todas as
coisas boas, e sua mãe, Fridolina, costumava levá"-la todo dia quando ia
visitar e consolar os doentes.

Quando Claribel foi para a escola, ficou ainda mais feliz, pois não
somente tinha sua casa como sempre a tivera, mas também muitos amigos
novos que eram da sua idade e veio a conhecer e amar. A~professora era
muito boa e muito gentil e muito velha, com um belo cabelo branco e um
rosto doce e gentil que nunca parecia duro ou sério, exceto quando
alguém contava uma mentira. Então, o sorriso desaparecia de seu rosto; e
era como a mudança no céu quando o sol se põe, e então ela ficava séria
e chorava silenciosamente. Se a criança que tinha sido malvada
confessasse o erro e prometesse nunca mais contar uma mentira, o sorriso
retornava como a luz do sol. Mas se a criança persistisse na mentira,
seu rosto se tornaria sério, e depois o aspecto sério ficaria na memória
do mentiroso, mesmo quando ela não estivesse presente.

Todo dia ela falava a todas as crianças sobre a beleza da Verdade e como
uma mentira era uma coisa muito obscura e terrível. Também lhes contava
histórias do Belo Livro; uma que ela amava, e que eles amavam também,
era sobre a Bela Cidade onde as pessoas boas vão viver depois daqui.

As crianças nunca se cansavam de ouvir sobre aquela Cidade, de uma
claridade cristalina como pedra de jaspe, com seus doze portões com os
nomes inscritos; e faziam perguntas à Professora sobre o Anjo que mediu
a Cidade com um junco dourado. Sempre perto do fim da história, a voz da
Professora se tornava muito séria, e um silêncio se entranhava nas
crianças, e elas ficavam mais juntas umas das outras, espantadas, quando
ela lhes contava que estaria condenado a ficar para sempre fora daquela
bela cidade, ``todo aquele que ama e conta uma mentira''.

Então, a boa Professora lhes contava que coisa terrível seria ficar ali
fora, e perder toda a beleza e a glória eterna que há lá dentro. E~tudo
por um erro que nenhum ser humano precisa cometer -- contar uma mentira.
Mesmo quando um erro foi cometido, as pessoas não ficam muito bravas se
a verdade é contada de uma vez; mas, se um erro é piorado por uma
mentira, então todo mundo fica bravo com razão. Se homens e mulheres,
até mesmo pais e mães que amam seus filhinhos com tanto afeto, ficam
bravos, o quanto mais bravo vai ficar Deus contra quem o pecado da
mentira é cometido?

Claribel amava essa história e muitas vezes chorava quando pensava nas
pobres pessoas que tinham de ficar fora da Bela Cidade para sempre, mas
ela nunca pensou que ela mesma iria contar uma mentira. Na verdade, ela
nunca pensou, até que veio a tentação. Quando as pessoas pensam muito
bem de si mesmas, perigam cometer um pecado, pois, se não ficarmos
sempre alertas contra o mal, certamente faremos algo errado; e como
Claribel não temia mal algum, podia ser facilmente levada ao pecado.

As crianças estavam todas ocupadas com suas contas de matemática.
Algumas delas sabiam aritmética, conseguiam suas respostas e
provavam"-nas; mas algumas não conseguiam a resposta certa, e outras
empacavam e não conseguiam qualquer resposta. Certas crianças levadas
nem mesmo tentavam chegar às respostas, mas faziam desenhos em suas
lousas e escreviam seus nomes. Claribel tentou resolver suas contas, mas
não conseguia lembrar 9 vezes 7, e ao invés de começar em ``duas vezes
um são dois'' e ir aumentando, ficou sem vontade e preguiçosa e desistiu
da conta, e fez começos de desenhos e desistiu deles também. Olhou para
a janela pensando em algo para desenhar e viu nos vidros mais baixos
flores coloridas, pintadas para impedir que as crianças olhassem para as
pessoas lá fora durante as lições. Claribel olhou fixo para uma dessas
flores, um lírio, e começou a desenhá"-lo.

Skooro viu"-a olhando e começou seu trabalho maléfico. Para ajudá"-la a
fazer o que não devia, ele tomou a forma de um lírio e apareceu na lousa
com traços bem fracos, de modo que ela só precisou desenhar em volta de
seus contornos, e então lá estava um lírio desenhado. Agora, não é
errado desenhar um lírio, e se Claribel o tivesse desenhado
apropriadamente, na hora certa, teria ganhado elogios; mas uma coisa boa
pode se tornar uma coisa má se for feita de modo errado -- e assim era
com o lírio de Claribel.

Depois de um instante, a Professora pediu as lousas. Quando Claribel
trouxe a dela, sabia que tinha feito algo errado e estava arrependida;
mas só estava arrependida porque estava com medo de ser punida. Quando a
Professora perguntou pelas respostas, ela baixou a cabeça e disse que
não tinha conseguido.

``Você tentou?'', perguntou a Professora.

``Sim'', ela respondeu, com a sensação de que tinha tentado por um
tempo.

``Ficou preguiçosa?'', perguntou de novo a professora. ``Você fez alguma
coisa além de suas contas?'' Então ela percebeu que, se contasse, teria
problemas por ter sido preguiçosa; e, então, esquecendo tudo sobre a
Cidade de Jaspe e aqueles que estão condenados a ficar fora de seus
belos portões, respondeu que não tinha feito mais nada a não ser as
contas. A~professora acreditou em sua palavra -- pois ela sempre fora
verdadeira -- e disse:

``Você ficou confusa, suponho, minha querida. Deixe"-me ajudá"-la'', e
gentilmente lhe mostrou como resolver as contas.

Quando estava voltando para seu assento, Claribel abaixou sua cabeça,
pois sabia que havia contado uma mentira, e, apesar de agora nunca
precisar ser descoberta, ficou triste e se sentiu como se estivesse fora
da Cidade cintilante. Bem naquele momento, se ela tivesse corrido para a
professora e tivesse dito ``Eu fiz uma coisa errada; mas serei uma
criança melhor da próxima vez'', tudo ficaria bem; mas ela não fez
assim, e a todo minuto que passava isso se tornava mais difícil de
fazer.

Logo depois a aula terminou, e Claribel foi triste para casa. Ela nem
pensou em brincar, pois havia contado uma mentira, e seu coração estava
pesaroso.

Quando chegou a hora de dormir, ela se deitou cansada, mas não conseguiu
dormir; e chorou amargamente, pois não conseguiu rezar. Estava
arrependida por ter contado uma mentira, e teve dificuldade em aceitar
que seu sofrimento não era suficiente para deixá"-la novamente feliz. Mas
sua consciência disse: ``Você não vai confessar amanhã?'' Mas ela pensou
que não seria necessário, pois o pecado já havia passado e ela não havia
feito mal a ninguém. Só que por todo o tempo sabia que havia feito uma
coisa errada. Se a professora tivesse falado sobre isso, teria dito: ``É
sempre assim, querida. Um pecado não pode ser expiado sem que a vergonha
tenha vindo primeiro; pois sem a vergonha e o reconhecimento da culpa, o
coração não pode ficar limpo do pecado''.

Finalmente, Claribel chorou até dormir.

Então, quando dormiu, a Criança Anjo entrou furtivamente no quarto e
passou por cima de suas pálpebras, de modo que até mesmo em seu sono ela
vira a bela luz, e pensou na Cidade como uma pedra jaspe, clara como um
cristal, com seus doze portões com os nomes inscritos. Sonhou que vira o
Anjo com o junco dourado medindo a cidade, e Claribel ficou tão feliz
que se esqueceu totalmente de seu pecado. A~Criança Anjo conhecia todos
os pensamentos dela, e ficou menor e menor até que toda a sua luz se
extinguiu. E~para Claribel, em seu sonho, tudo pareceu ficar enegrecido,
e ela percebeu que estava do lado de fora do portão da Bela Cidade. O~Anjo, que segurava o junco dourado de medir, estava nas muralhas da
cidade, e, com uma voz terrível, disse:

``Claribel, deves ficar do lado de fora; tu contas e ama uma mentira''.

``Oh, não'', disse Claribel, ``Não a amo''.

``Então por que não confessas teu erro?''

Claribel calou"-se. Mas ela não iria confessar seu pecado, pois seu
coração estava endurecido; o Anjo levantou seu junco dourado e, veja!,
fez brotar um belo lírio. Então, o Anjo disse:

``Os lírios crescem somente para os puros, que vivem dentro da cidade;
tu deves ficar aqui fora entre os mentirosos''.

Claribel viu os muros de jaspe diante de si se elevando cada vez mais
alto, e soube que eles eram uma barreira eterna e que deveria para
sempre ficar do lado de fora da Bela Cidade. E, em angústia e horror,
ela sentiu como era profundo o seu pecado, e ansiou confessá"-lo.

Skooro viu que ela estava se arrependendo, pois ele, também, podia ver
seus pensamentos, e com a escuridão de sua presença tentou apagar todo o
sonho da Bela Cidade.

Mas a Criança Anjo infiltrou"-se em seu coração e deixou"-o leve; e a
semente da penitência cresceu e floresceu.

Claribel acordou cedo, levantou"-se, e contou à sua professora seu
pecado, e ficou feliz mais uma vez.

Por toda sua vida ela amou os lírios, pois lembrava de sua mentira e de
seu arrependimento, e que os lírios crescem dentro da Cidade Jaspe, que
é somente para os puros.
